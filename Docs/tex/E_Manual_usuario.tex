\apendice{Documentación de usuario}
\section{Introducción}
Para acceder como usuario a esta aplicación hará falta tener acceso al servidor de la UBU en el espacio \textit{www3.ubu.es/boletincoyuntura}
\section{Manual del usuario}
\subsection{Acceso a la aplicación}
\imagen{imagenes/inicio}{Pantalla de inicio de la aplicación}
Desde esta pantalla se podrá acceder a la aplicación como invitando seleccionando \textit{Comenzar}, solicitar el registro accediendo a \textit{Registro} o entrar a la aplicación si ya somos usuarios en el apartado \textit{Iniciar sesión}.\\
\subsection{Registro}
\imagen{imagenes/registro}{Formulario para el registro de la aplicación}
\subsection{Iniciar sesión}
\imagen{imagenes/login}{Formulario de inicio de sesión}
\subsection{Gestión de datos del INE}
Se podrá elegir entre subir datos o actualizar las tablas de la base de datos.
\imagen{imagenes/menuINE}{Menú de la gestión de datos del INE}
\subsubsection{Seleccionar dirección}
\imagen{imagenes/direccionine}{Pantalla para introducir la dirección de la página del INE}
En esta pantalla se pondrá la dirección de la tabla del INE que se quiera insertar en la base de datos.
\subsubsection{Elegir datos}
\imagen{imagenes/elegirDatosINE}{Selección de los conjuntos de datos}
\subsubsection{Subir datos}
\imagen{imagenes/subirdatos}{Pantalla para subir los datos}
En esta pantalla se podrá seleccionar a qué variable, ámbito y categoría de los existentes pertenecen los datos o crear una nueva tabla donde guardar los datos.
\subsubsection{Actualizar datos}
\subsection{Tablas predefinidas}
Las tablas predefinidas son las variables almacenadas en el sistema, pero asignados los valores que corresponden a cada categoría en sus determinados ámbitos geográficos y años. En este apartado se muestran todas las variables.
\imagen{imagenes/mostrarVariables}{Tablas predefinidas}
\subsubsection{Ver tabla} 
Primero se deben seleccionar los valores que se quieran ver en el formulario y luego se podrá visualizar la tabla.\\
Desde esta pantalla también se puede exportar a xls la información seleccionada.
\imagen{imagenes/filtrarDatos}{Seleccionar datos}
Una vez rellenado el formulario y enviados los datos, las tablas se mostrarán de la siguiente manera.
\imagen{imagenes/verTabla}{Tabla}
Desde esta pantalla también se pueden ver los datos de la tabla en forma de gráfico.
\imagen{imagenes/verGrafico}{Filtrar los datos del gráfico}
\imagen{imagenes/verGrafico2}{Ver gráfico}
\subsubsection{Crear tablas}
\imagen{imagenes/crearTabla}{Formulario para crear una nueva tabla}
\imagen{imagenes/insertarTabla}{Formulario para rellenar los valores de la tabla}
\subsubsection{Modificar tablas}
\imagen{imagenes/modificartabla}{Pantalla para modificar las tablas}
Si se seleciona la pestaña Opciones, se podrá seleccionar si se quiere añadir o borrar un ámbito, una categoría o un año de la tabla.
\subsubsection{Añadir ámbito}
\imagen{imagenes/añadirambito}{Pantalla para añadir un ámbito}
Desde aquí se podrá añadir un ámbito nuevo a la tabla e introducir los valores asociados.
\subsubsection{Borrar ámbito}
\imagen{imagenes/borrarambito}{Pantalla para borrar los ámbitos que se deseen}

\subsubsection{Añadir año}
\imagen{imagenes/añadiraño}{Pantalla para añadir un año}
Desde aquí se podrá añadir un año nuevo a la tabla e introducir los valores asociados.
\subsubsection{Borrar año}
\imagen{imagenes/borraraño}{Pantalla para borrar los años que se deseen}

\subsubsection{Añadir categoría}
\imagen{imagenes/añadircategoria}{Pantalla para añadir una categoría}
Desde aquí se podrá añadir una categoría nueva a la tabla e introducir los valores asociados.
\subsubsection{Borrar categoría}
\imagen{imagenes/borrarcategoria}{Pantalla para borrar una categoría}

\subsection{Gestión de datos}
\imagen{imagenes/menugestionvariables}{Menú de la gestión de los datos}
\subsubsection{Mostrar variables}

\imagen{imagenes/mostrarVariables}{Variables del sistema}
\subsubsection{Modificar variable}
\imagen{imagenes/modificarVariables}{Pantalla para modificar las variables}

\subsubsection{Mostrar súper categorías}
\imagen{imagenes/versupercategorias}{Súper categorías del sistema}
\subsubsection{Crear nueva súper categoría}
Al crear una nueva súper categoría es posible asignarle categorías del sistema que no posean una.
\imagen{imagenes/crearsupercategoria}{Pantalla para crear una nueva súper categoría}
\subsubsection{Modificar súper categorías}
Al modificar una súper categoría se le puede cambiar el nombre y las categorías que tenga asignadas.
\imagen{imagenes/modificarsupercategorias}{Modificar una súper categoría}

\subsubsection{Mostrar categorías}
\imagen{imagenes/vercategoria}{Categorías del sistema}
\subsubsection{Crear nueva categoría}
Al crear una nueva categoría se le puede asignar una súper categoría.
\imagen{imagenes/crearcategoria}{Pantalla para crear una nueva categoría}
\subsubsection{Modificar categorías}
\imagen{imagenes/modificarcategoria}{Modificar una categoría}

\subsubsection{Mostrar ámbitos}
\imagen{imagenes/verambitos}{Ámbitos del sistema}
\subsubsection{Crear nuevo ámbito}
\imagen{imagenes/crearambito}{Pantalla para crear un nuevo ámbito}
\subsubsection{Modificar ámbito}
\imagen{imagenes/modificarambito}{Modificar un ámbito}

