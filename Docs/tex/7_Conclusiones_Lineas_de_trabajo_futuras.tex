\capitulo{7}{Conclusiones y Líneas de trabajo futuras}
\section{Conclusiones técnicas}
Considero que casi todos los objetivos impuestos se han cumplido satisfactoriamente. Los errores de la aplicación se han solucionado y además se han conseguido añadir las nuevas funcionalidades propuestas:
\begin{itemize}
    \item Extracción de datos desde el INE.
    \item Predicción de datos y representación gráfica.
    \item Edición de usuarios.
    \item Extracción de las tablas a formato Excel.
\end{itemize}
Además se ha cumplido una de las principales exigencias por la parte del cliente que fue la subida de la aplicación al servidor de la UBU.
El objetivo que cumplí desde el principio del trabajo fue el seguimiento con la herramienta Github. Tuve muchos problemas para configurar esta herramienta con mi repositorio y con las otras aplicaciones de integración continua que he usado.
\section{Conclusiones personales}
Puedo concluir diciendo que este ha sido un proyecto que me ha supuesto mucho esfuerzo porque no había trabajado nunca con este entorno de desarrollo, ni este lenguaje ni sabía nada sobre el desarrollo y despliegue de aplicaciones web.\\
Aunque hubo un tiempo en el que me arrepentí de haber elegido un tema del que no tenía conocimientos previos, tengo que reconocer que he aprendido mucho y esta experiencia me ha enseñado a ser autodidacta y a cómo afrontar proyectos futuros en solitario sin nadie que me guíe.
\section{Líneas de trabajo futuras}
Tengo algunas ideas sobre posibles cambios o mejoras respecto al proyecto.
\begin{itemize}
    \item Mejorar la extracción de datos desde el INE ya que hay algunos conjuntos de datos que no es capaz de reconocer.
    \item Migrar la aplicación a una app móvil.
    \item Importar datos desde otras plataformas mencionadas anteriormente como Eurostat o los datos abiertos de la Junta de Castilla y León.
    \item Sería interesante poder exportar todos los datos de las tablas y de los gráficos a otro tipo de ficheros como por ejemplo “.pdf”,“.doc”, etc.
\end{itemize}