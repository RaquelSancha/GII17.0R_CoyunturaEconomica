\capitulo{4}{Técnicas y herramientas}
\section{Técnicas}
\subsection{Scrum}
En Scrum un proyecto se ejecuta en ciclos temporales cortos y de duración fija (iteraciones o sprints).\cite{scrum:definicion}\\
A cada sprint se le asigna unas tareas, generalmente con relación entre sí y en torno a un mismo tema.\\ 
Al finalizar la iteración se revisan las tareas en una reunión con el resto de miembros del equipo. 
\subsection{Integración continua}
La integración continua es una práctica de ingeniería de software que consiste en hacer integraciones automáticas de un proyecto lo más a menudo posible para así poder detectar fallos cuanto antes. Entendemos por integración la compilación y ejecución de pruebas de todo un proyecto.\cite{IntegracionContinua}
Ventajas de usar la integración continua 
\begin{itemize}
    \item Capacidad de detección de problemas temprana, lo que facilita su solución.
    \item Disponibilidad en cualquier momento de distintas versiones.
    \item Ejecución inmediata de las pruebas unitarias.
    \item Monitorización continua de las métricas de calidad del código del proyecto.
\end{itemize}
\section{Herramientas}
Voy a clasificar las herramientas según las fases en las que han sido utilizadas.  
\subsection{Desarrollo}
\subsubsection{Laravel}
El framework usado para el desarrollo de la aplicación ha sido Laravel.\\
Laravel incluye Eloquent, un mapeador relacional de objetos (ORM) que simplifica la interacción con la base de datos. Cuando se usa Eloquent, cada tabla de la base de datos tiene un modelo correspondiente que se usa para interactuar con esa tabla. Además de recuperar registros de la tabla de la base de datos, los modelos Eloquent también le permiten insertar, actualizar y eliminar registros de la tabla.\cite{Laravel:modelos}\\
Laravel también se caracteriza por el uso de controladores para agrupar la funcionalidad de un determinado recurso.\cite{Laravel:controladores}\\
Posee una estructura de directorios predefinidos que ayuda a organizar nuestro proyecto. Los más importantes son \textit{resources} que contiene las vistas de la aplicación, \textit{routes} que tiene el archivo que guarda la definición de las rutas y \textit{app} que guarda los modelos y los controladores entre otros archivos de configuración del proyecto.\\

\subsubsection{Composer}
Composer es un sistema de gestión de paquetes para programar en PHP el cual provee los formatos estándar necesarios para manejar dependencias y librerías de PHP. \cite{Composer:definicion}\\
Es un gestor de dependencias para proyectos escritos en el leguaje de programación PHP. Eso quiere decir que nos permite gestionar (declarar, descargar y mantener actualizados) los paquetes de software en los que se basa nuestro proyecto PHP.\\
Cuando se empieza un proyecto en PHP, ya de cierta complejidad, no  vale solo con la librería de funciones nativa de PHP. Generalmente se usa alguna que otra librería de terceros desarrolladores, que permite evitar empezar todo desde cero. Ya sea un framework o algo más específico como un sistema para debug o envío de email, registro de usuarios, exportación de datos, etc., cualquier cosa que se pueda necesitar ya puede estar creada por otros desarrolladores.\\
Para usar composer debemos tener un archivo JSON en el que deberemos escribir los paquetes que queramos instalar, el nombre del proyecto, la descripción, algunos comandos que queramos que se ejecuten en el momento de la instalación o de la actualización (como por ejemplo generar la clave del proyecto u optimizarlo), etc. Puede haber muchas opciones de configuración posibles en este archivo. 
Este archivo debe llamarse composer.json.\\
Al usar el comando \textit{composer install}, se crea automáticamente un archivo llamado \textit{composer.lock} donde aparece una información más detallada de las dependencias instaladas además de ir a los repositorios de paquetes de software y descargar aquellas librerías mencionadas, copiándolas en la carpeta del proyecto.\\
Estas dependencias se instalan en la carpeta \textit{vendor} de la aplicación.\cite{composer}
\subsubsection{Visual Studio Code}
Es un entorno de desarrollo que permite crear sitios y aplicaciones web. Es compatible con múltiples lenguajes de programación, tales como C++, C\#, Visual Basic .NET, F\#, Java, Python, Ruby y PHP (Éste último es el que se ha usado para el proyecto).\cite{VisualStudio} 
Esta herramienta detecta el lenguaje, da sugerencias de escritura, detecta errores y proporciona posibles soluciones. También se puede sincronizar con el proyecto en Github facilitando el seguimiento del proyecto.
\subsubsection{Git}
Se ha usado Git como herramienta para el control de las versiones del proyecto. Además se puede usar junto con Github desde Visual Studio Code descargándonos su extensión.

\subsection{Despliegue}

\subsubsection{Xampp}
XAMPP es un paquete de software libre, que consiste principalmente en una pila de herramientas de desarrollo que incluye un sistema de gestión de bases de datos MySQL, el servidor web Apache y los intérpretes para lenguajes de script PHP y Perl. El nombre es en realidad un acrónimo: X (para cualquiera de los diferentes sistemas operativos), Apache, MariaDB/MySQL, PHP, Perl.\cite{Xampp}
He usado esta aplicación para las pruebas locales de mi proyecto.
\subsubsection{Docker}
Docker sirve para automatizar el despliegue de aplicaciones dentro de contenedores software de forma que se pueda probar la aplicación en distintos equipos sin la necesidad de instalar un servidor local como Xampp.\\ \cite{Docker}
Docker tiene como ventaja que aísla solo los recursos del sistema operativo que necesita y no una cantidad de recursos fija como hace una máquina virtual.\\
Para el despliegue de mi aplicación he utilizado un contenedor especial para aplicaciones de Laravel llamado Laradock.\\
Laradock contiene multitud de imágenes conectadas entre ellas que se usan para el despliegue de la app. Sin embargo, solo se han usado Apache, PHP, PhpMyAdmin, MySQL y Selenium.\\
\subsubsection{Heroku}
Heroku es una plataforma en la nube que permite subir aplicaciones para probarlas. La principal ventaja que tiene Heroku es que su uso a un nivel básico es gratuito. Como desventaja diría que la aplicación va muy lenta en este servidor web.
\subsection{Calidad del código}
\subsubsection{Codacy}
Para evaluar la calidad del código se ha usado la herramienta Codacy.\\
Permite realizar un análisis del proyecto y reportar los posibles problemas que haya. En el apartado \textit{Issues} se pueden ver clasificados por varios criterios: 
\begin{itemize}
    \item Lenguaje de programación.
    \item Categorías: Estilo del código, seguridad y código sin usar.
    \item Nivel de peligrosidad: Distingue entre errores y warnings.
    \item Patrones no recomendados: Accesos estáticos, nombres de variables cortos, etc.
\end{itemize}
En el apartado \textit{Files} aparecen los archivos del proyecto calificados con una nota en función del número de \textit{Issues} que tenga, el grado de duplicación y la complejidad.
También permite ver la calidad de los cambios desde el apartado \textit{Commits}.\\
Esta aplicación es muy fácil de usar y se puede conectar a un repositorio de Github de manera que al hacer algún cambio en éste el análisis se actualice.\\
 \imagen{imagenes/codacy}{Opciones que ofrece codacy para el análisis de nuestro código}
\subsection{Documentación}
\subsubsection{Latex}
\LaTeX{} es una herramienta para la composición de texto con una serie de comandos que permiten formar documentos a gusto del usuario.\\
Está formada por un gran conjunto de macros de TeX, con la intención de facilitar el uso del lenguaje de composición tipográfica.\cite{wiki:latex}\\
Su documentación es muy amplia y detallada con multitud de ejemplos.
\section{Librerías}
\subsection{Laravel Excel}
Para exportar los datos de las tablas a Excel se ha usado la biblioteca Laravel Excel.
Esta biblioteca está basada en PhpSpreadsheet que es un recurso escrito en PHP puro el cual proporciona un conjunto de clases que permiten leer y escribir en diferentes formatos de archivo de hoja de cálculo, como Excel y LibreOffice Calc.\cite{LaravelExcel}\\

\subsection{Chart js}
Esta librería en JavaScript es gratuita y de código abierto y se usa para la visualización de datos en forma de gráfico.
Tipos de gráficos que permite crear:
\begin{itemize}
    \item Gráfico de barras.
    \item Gráfico de líneas.
    \item Área.
    \item Burbuja.
    \item Radar.
    \item Polar.
    \item Dispersión.
\end{itemize}
\subsection{Laravel DebugBar}
Este proyecto, realizado por \href{https://github.com/barryvdh}{un usuario de github}, consiste en una DebugBar elaborada exclusivamente para proyectos en Laravel.\\
Ayuda a identificar errores y a recopilar información de ejecución de la aplicación. Algunos de los datos que recopila son:
\begin{itemize}
    \item Resultado de las consultas realizadas a la base de datos.
    \item Información de la ruta en la que estamos.
    \item Información de las vistas cargadas.
    \item Eventos.
    \item Información de la versión de Laravel y del entorno de desarrollo.
    \item Datos de los usuarios.
    \item Valores de los archivos de configuración.
\end{itemize}
\imagen{imagenes/debugbar}{DebugBar para Laravel}
\subsection{Rubix ML}
Esta biblioteca ha sido usada para la predicción de los datos. Incluye múltiples algoritmos de aprendizaje tanto para datos categóricos como continuos. En el caso de mi aplicación he usado los algoritmos para datos continuos.\\
Las fases por las que tienen que pasar los datos para realizar la predicción son:
\begin{itemize}
    \item Extracción.
    \item Transformación.
    \item Carga.
    \item Entrenamiento.
    \item Predicción.
\end{itemize}
Para realizar la estimación, hay que precargar los datos que tenemos, adaptarlos y pasárselos al estimador para entrenarlo. Los posibles algoritmos que se usan para la estimación los he explicado en el apartado de conceptos teóricos.
