\apendice{Especificación de Requisitos}

\section{Introducción}
En esta sección hablaré de los requisitos que cumple el proyecto en su totalidad aunque haré más hincapié en los que he creado yo.\\
Al no ser mi propio proyecto sino que he mejorado uno existente, muchos de los requisitos ya estaban implementados pero me ha parecido interesante mencionarlos y hablar de ellos para que el proyecto se entienda mejor. Además he mejorado o arreglado muchos de ellos.\\
Para poder diferenciar entre los que ya estaban en la aplicación y los que he creado yo los señalaré .\\
Es importante señalar que existen tres roles:
\begin{description}
    \item [Súper administrador] Tiene acceso a todas las funcionalidades del sistema. Se diferencia del administrador normal en que éste no tiene acceso a la gestión de los usuarios.
    \item [Administrador] Tiene acceso a todas las funcionalidades menos a la de gestión de los usuarios.
    \item [Invitado] Sólo se le permite acceder a la información de las tablas ya definidas y a la ayuda de la aplicación. Serán las personas que accedan a la aplicación a través del navegador.
\end{description}
\section{Objetivos generales}
Los objetivos generales del proyecto son:
\begin{itemize}
    \item Detectar y corregir los fallos del proyecto anterior.
    \item Permitir la entrada de datos del Instituto nacional de estadística (INE) de una forma más sencilla para el usuario.
    \item Mejorar la seguridad de la aplicación encriptando las contraseñas.
    \item Realizar una predicción de los datos tomando de referencia los existentes en la base de datos.
    \item Mejorar la ayuda de la aplicación para facilitar al usuario su utilización.
\end{itemize}
\section{Catalogo de requisitos}
A continuación, se enumeran los requisitos específicos derivados de los objetivos generales del proyecto.
\subsection{Requisitos funcionales}
\begin{description}
    \item [RF1 Registro en la aplicación web] Los nuevos usuarios podrán solicitar el registro en la aplicación por medio de un formulario con los siguientes campos: Nombre, correo electrónico y contraseña. Posteriormente el súper administrador aceptará la solicitud.
    \item [RF2 Login] Acceso a la aplicación.
    \item [RF3 Gestión de tablas] Gestión de las tablas de la aplicación.
    \begin{description}
        \item [RF3.1 Inserción de tablas] Insertar nuevas tablas a la base de datos.
        \item [RF3.2 Creación de tablas con categorías de cualquier variable] La aplicación podrá crear tablas con cualquier categoría de cualquier variable.
        \item [RF3.3 Modificación de tablas] Se permitirá modificar los valores que forman parte de una tabla.
        \begin{description}
            \item [RF3.3.1 Añadir año]
            \item [RF3.3.2 Añadir categoría] Se podrá añadir categorías nuevas y se podrá asignar la súper categoría a la que pertenecen.
            \item [RF3.3.3 Añadir ámbito geográfico] 
            \item [RF3.3.4 Borrar año]
            \item [RF3.3.5 Borrar categoría]
            \item [RF3.3.6 Borrar ámbito geográfico]
        \end{description}
        \item [RF3.4 Borrado de tablas]
    \end{description}
    \item [RF4 Creación de gráficos] Se generarán gráficos automáticamente a partir de los datos introducidos en la tabla, se podrá filtrar los datos que se desean ver en el gráfico por medio de un formulario.
    \item [RF5 Gestión de datos] Gestión de los datos de la aplicación.
    \begin{description}
        \item [RF5.1 Variables] Gestión de las variables.
        \begin{description}
            \item [RF5.1.1 Modificar] Se podrá modificar el nombre, la fuente, el tipo y la descripción.
            \item [RF5.1.2 Borrar] Se borrará una variable con todos los datos asociados a ella.
        \end{description}
        \item [RF5.2 Categorías] Gestión de las categorías.
        \begin{description}
            \item [RF5.2.1 Crear] Se creará una categoría y se la podrá asignar a una súper categoría, si no se selecciona ninguna, se introducirá a la súper categoría “Sin categoría”.
            \item [RF5.2.2 Modificar] Se podrá modificar el nombre y la súper categoría a la que está asignada una categoría.
            \item [RF5.2.3 Borrar]  Se podrá elegir si quieres borrar una categoría del sistema o tan solo de una variable.
        \end{description}
        \item [RF5.3 Súper categorías] Gestión de las súper categorías.
        \begin{description}
            \item [RF5.3.1 Crear] Se creará una súper categoría nueva y se le podrá asignar las categorías que están sin categoría.
            \item [RF5.3.2 Modificar] Se podrá modificar el nombre y las categorías que están asignadas.
            \item [RF5.3.3 Borrar] Borrar una súper categoría del sistema, si la súper categoría tenía categorías estas pasarán a “Sin categoría”.
        \end{description}
        \item [RF5.4 Ámbitos geográficos] Gestión de los ámbitos geográficos.
        \begin{description}
            \item [RF5.4.1 Crear] Se podrá crear un ámbito geográfico nuevo.
            \item [RF5.4.2 Modificar] Se modificará el nombre de un ámbito geográfico.
            \item [RF5.4.3 Borrar] Se podrá elegir si se desea borrar un ámbito del sistema o tan solo de una variable.
        \end{description}
    \end{description}
    \item [RF6 Exportación de tablas a Excel] La aplicación puede exportar las tablas creadas a partir de los datos almacenados en la base de datos a ficheros “.xls”.
    \item[RF7 Gestión de usuarios] Sólo tiene acceso a esta funcionalidad el súper administrador.
    \begin{description}
        \item[RF7.1 Modificar rol de los usuarios] Se podrá cambiar el rol de los usuarios de administrador a súper administrador.
        \item[RF7.2 Borrar usuario] Borrar usuario del sistema.
        \item[RF7.3 Gestión de peticiones de nuevos usuarios] Se podrá aceptar o declinar una petición de registro en la aplicación.
        \begin{description}
            \item[RF7.3.1 Aceptar petición] El usuario pasa a la base de datos como usuario de la aplicación.
            \item[RF7.3.2 Declinar petición] Se declina su solicitud y se borra de la tabla de usuarios por confirmar.
        \end{description}
    \end{description}
\end{description}
\section{Especificación de requisitos}


