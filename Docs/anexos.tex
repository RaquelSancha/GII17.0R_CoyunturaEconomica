\documentclass[a4paper,12pt,twoside]{memoir}

% Castellano
\usepackage[spanish,es-tabla]{babel}
\selectlanguage{spanish}
\usepackage[utf8]{inputenc}
\usepackage[T1]{fontenc}
\usepackage{lmodern} % scalable font
\usepackage{microtype}
\usepackage{placeins}

\RequirePackage{booktabs}
\RequirePackage[table]{xcolor}
\RequirePackage{xtab}
\RequirePackage{multirow}

% Links
\usepackage[colorlinks]{hyperref}
\hypersetup{
	allcolors = {red}
}

% Ecuaciones
\usepackage{amsmath}

% Rutas de fichero / paquete
\newcommand{\ruta}[1]{{\sffamily #1}}

% Párrafos
\nonzeroparskip


% Imagenes
\usepackage{graphicx}
\newcommand{\imagen}[2]{
	\begin{figure}[!h]
		\centering
		\includegraphics[width=0.9\textwidth]{#1}
		\caption{#2}\label{fig:#1}
	\end{figure}
	\FloatBarrier
}

\newcommand{\imagenflotante}[2]{
	\begin{figure}%[!h]
		\centering
		\includegraphics[width=0.9\textwidth]{#1}
		\caption{#2}\label{fig:#1}
	\end{figure}
}



% El comando \figura nos permite insertar figuras comodamente, y utilizando
% siempre el mismo formato. Los parametros son:
% 1 -> Porcentaje del ancho de página que ocupará la figura (de 0 a 1)
% 2 --> Fichero de la imagen
% 3 --> Texto a pie de imagen
% 4 --> Etiqueta (label) para referencias
% 5 --> Opciones que queramos pasarle al \includegraphics
% 6 --> Opciones de posicionamiento a pasarle a \begin{figure}
\newcommand{\figuraConPosicion}[6]{%
  \setlength{\anchoFloat}{#1\textwidth}%
  \addtolength{\anchoFloat}{-4\fboxsep}%
  \setlength{\anchoFigura}{\anchoFloat}%
  \begin{figure}[#6]
    \begin{center}%
      \Ovalbox{%
        \begin{minipage}{\anchoFloat}%
          \begin{center}%
            \includegraphics[width=\anchoFigura,#5]{#2}%
            \caption{#3}%
            \label{#4}%
          \end{center}%
        \end{minipage}
      }%
    \end{center}%
  \end{figure}%
}

%
% Comando para incluir imágenes en formato apaisado (sin marco).
\newcommand{\figuraApaisadaSinMarco}[5]{%
  \begin{figure}%
    \begin{center}%
    \includegraphics[angle=90,height=#1\textheight,#5]{#2}%
    \caption{#3}%
    \label{#4}%
    \end{center}%
  \end{figure}%
}
% Para las tablas
\newcommand{\otoprule}{\midrule [\heavyrulewidth]}
%
% Nuevo comando para tablas pequeñas (menos de una página).
\newcommand{\tablaSmall}[5]{%
 \begin{table}
  \begin{center}
   \rowcolors {2}{gray!35}{}
   \begin{tabular}{#2}
    \toprule
    #4
    \otoprule
    #5
    \bottomrule
   \end{tabular}
   \caption{#1}
   \label{tabla:#3}
  \end{center}
 \end{table}
}

%
%Para el float H de tablaSmallSinColores
\usepackage{float}

%
% Nuevo comando para tablas pequeñas (menos de una página).
\newcommand{\tablaSmallSinColores}[5]{%
 \begin{table}[H]
  \begin{center}
   \begin{tabular}{#2}
    \toprule
    #4
    \otoprule
    #5
    \bottomrule
   \end{tabular}
   \caption{#1}
   \label{tabla:#3}
  \end{center}
 \end{table}
}

\newcommand{\tablaApaisadaSmall}[5]{%
\begin{landscape}
  \begin{table}
   \begin{center}
    \rowcolors {2}{gray!35}{}
    \begin{tabular}{#2}
     \toprule
     #4
     \otoprule
     #5
     \bottomrule
    \end{tabular}
    \caption{#1}
    \label{tabla:#3}
   \end{center}
  \end{table}
\end{landscape}
}

%
% Nuevo comando para tablas grandes con cabecera y filas alternas coloreadas en gris.
\newcommand{\tabla}[6]{%
  \begin{center}
    \tablefirsthead{
      \toprule
      #5
      \otoprule
    }
    \tablehead{
      \multicolumn{#3}{l}{\small\sl continúa desde la página anterior}\\
      \toprule
      #5
      \otoprule
    }
    \tabletail{
      \hline
      \multicolumn{#3}{r}{\small\sl continúa en la página siguiente}\\
    }
    \tablelasttail{
      \hline
    }
    \bottomcaption{#1}
    \rowcolors {2}{gray!35}{}
    \begin{xtabular}{#2}
      #6
      \bottomrule
    \end{xtabular}
    \label{tabla:#4}
  \end{center}
}

%
% Nuevo comando para tablas grandes con cabecera.
\newcommand{\tablaSinColores}[6]{%
  \begin{center}
    \tablefirsthead{
      \toprule
      #5
      \otoprule
    }
    \tablehead{
      \multicolumn{#3}{l}{\small\sl continúa desde la página anterior}\\
      \toprule
      #5
      \otoprule
    }
    \tabletail{
      \hline
      \multicolumn{#3}{r}{\small\sl continúa en la página siguiente}\\
    }
    \tablelasttail{
      \hline
    }
    \bottomcaption{#1}
    \begin{xtabular}{#2}
      #6
      \bottomrule
    \end{xtabular}
    \label{tabla:#4}
  \end{center}
}

%
% Nuevo comando para tablas grandes sin cabecera.
\newcommand{\tablaSinCabecera}[5]{%
  \begin{center}
    \tablefirsthead{
      \toprule
    }
    \tablehead{
      \multicolumn{#3}{l}{\small\sl continúa desde la página anterior}\\
      \hline
    }
    \tabletail{
      \hline
      \multicolumn{#3}{r}{\small\sl continúa en la página siguiente}\\
    }
    \tablelasttail{
      \hline
    }
    \bottomcaption{#1}
  \begin{xtabular}{#2}
    #5
   \bottomrule
  \end{xtabular}
  \label{tabla:#4}
  \end{center}
}



\definecolor{cgoLight}{HTML}{EEEEEE}
\definecolor{cgoExtralight}{HTML}{FFFFFF}

%
% Nuevo comando para tablas grandes sin cabecera.
\newcommand{\tablaSinCabeceraConBandas}[5]{%
  \begin{center}
    \tablefirsthead{
      \toprule
    }
    \tablehead{
      \multicolumn{#3}{l}{\small\sl continúa desde la página anterior}\\
      \hline
    }
    \tabletail{
      \hline
      \multicolumn{#3}{r}{\small\sl continúa en la página siguiente}\\
    }
    \tablelasttail{
      \hline
    }
    \bottomcaption{#1}
    \rowcolors[]{1}{cgoExtralight}{cgoLight}

  \begin{xtabular}{#2}
    #5
   \bottomrule
  \end{xtabular}
  \label{tabla:#4}
  \end{center}
}




\graphicspath{ {./img/} }

% Capítulos
\chapterstyle{bianchi}
\newcommand{\capitulo}[2]{
	\setcounter{chapter}{#1}
	\setcounter{section}{0}
	\chapter*{#2}
	\addcontentsline{toc}{chapter}{#2}
	\markboth{#2}{#2}
}

% Apéndices
\renewcommand{\appendixname}{Apéndice}
\renewcommand*\cftappendixname{\appendixname}

\newcommand{\apendice}[1]{
	%\renewcommand{\thechapter}{A}
	\chapter{#1}
}

\renewcommand*\cftappendixname{\appendixname\ }

% Formato de portada
\makeatletter
\usepackage{xcolor}
\newcommand{\tutor}[1]{\def\@tutor{#1}}
\newcommand{\course}[1]{\def\@course{#1}}
\definecolor{cpardoBox}{HTML}{E6E6FF}
\def\maketitle{
  \null
  \thispagestyle{empty}
  % Cabecera ----------------
\noindent\includegraphics[width=\textwidth]{cabecera}\vspace{1cm}%
  \vfill
  % Título proyecto y escudo informática ----------------
  \colorbox{cpardoBox}{%
    \begin{minipage}{.8\textwidth}
      \vspace{.5cm}\Large
      \begin{center}
      \textbf{TFG del Grado en Ingeniería Informática}\vspace{.6cm}\\
      \textbf{\LARGE\@title{}}
      \end{center}
      \vspace{.2cm}
    \end{minipage}

  }%
  \hfill\begin{minipage}{.20\textwidth}
    \includegraphics[width=\textwidth]{escudoInfor}
  \end{minipage}
  \vfill
  % Datos de alumno, curso y tutores ------------------
  \begin{center}%
  {%
    \noindent\LARGE
    Presentado por \@author{}\\ 
    en Universidad de Burgos --- \@date{}\\
    Tutor: \@tutor{}\\
  }%
  \end{center}%
  \null
  \cleardoublepage
  }
\makeatother


% Datos de portada
\title{título del TFG \\Documentación Técnica}
\author{nombre alumno}
\tutor{nombre tutor}
\date{\today}

\begin{document}

\maketitle



\cleardoublepage



%%%%%%%%%%%%%%%%%%%%%%%%%%%%%%%%%%%%%%%%%%%%%%%%%%%%%%%%%%%%%%%%%%%%%%%%%%%%%%%%%%%%%%%%



\frontmatter


\clearpage

% Indices
\tableofcontents

\clearpage

\listoffigures

\clearpage

\listoftables

\clearpage

\mainmatter

\appendix

\apendice{Plan de Proyecto Software}

\section{Introducción}
En este apartado se va a hablar de la planificación temporal del proyecto mediante los sprints de github que es la herramienta que se ha usado para organizar las tareas y su secuencia de ejecución.\\
También se hará un estudio sobre la viabilidad del proyecto para comprobar que su desarrollo en un marco real como puede ser el de una empresa sería posible.
\section{Planificación temporal}
Se ha intentado usar la metodología SCRUM pero adaptándola a mi forma de trabajar y a mis capacidades. Con esto quiero decir que aunque se ha trabajado con sprints, éstos no han tenido una duración fija y preestablecida como dice que deben ser en la definición de la técnica.\\
Estas iteraciones han tenido unas duraciones variables, de 15 días hasta 2 meses, dependiendo de las tareas de dichos sprints y de la dificultad que me ha supuesto resolverlas.\\
Se realizaban reuniones de revisión al finalizar cada sprint y se pensaban y preparaban las siguientes tareas a realizar.
Estas tareas se estimaban y priorizaban con la ayuda del tablón que nos ofrece Github.
\imagen{imagenes/tablonGithub}{Organización de las tareas}
Para monitorizar el progreso del proyecto se han utilizado los gráficos burndown que ofrece la extensión ZenHub.
\subsection{Primer sprint: Inicio del proyecto}
En este primer sprint las principales tareas fueron:
\begin{itemize}
    \item Importación del proyecto anterior
    \item Leer su documentación.
    \item Descargar las herramientas necesarias para su producción.
    \item Probar el proyecto e identificar los fallos.
    \item Aprendizaje del lenguaje de programación y del entorno del trabajo.
\end{itemize}
\imagen{imagenes/sprint1}{Gráfica burndown del sprint 1}
\subsection{Segundo sprint: Inicio del proyecto parte 2}
Este sprint fue el más difícil para mí porque todavía no estaba familiarizada del todo con el código y me costó mucho empezar a programar.\\
Las tareas de este sprint fueron ejecutar los tests que ya estaban hechos para probar el proyecto y empezar a arreglar los fallos que había en la administración de usuarios.\\
\imagen{imagenes/sprint2}{Gráfica burndown del sprint 2}
\subsection{Tercer sprint: Fase de pruebas}
En esta iteración realicé una serie de tests para poner a prueba a la aplicación y así detectar sus posibles errores para posteriormente corregirlos.\\
Para ello usé el proyecto de barryvdh consistente en una debugbar para laravel.\\
\imagen{imagenes/debugbar}{Debugbar para Laravel}
\imagen{imagenes/sprint3}{Gráfica burndown del sprint 3}
\subsection{Cuarto sprint:  Primera fase de cambios en el proyecto}
El sprint se llama así porque es cuando realmente empecé a hacer cambios significativos en el proyecto.
Las principales tareas que se llevaron a cabo fueron:
\begin{itemize}
    \item Crear las migraciones de las tablas de la base de datos.
    \item Empezar a pensar la funcionalidad de la extracción de datos desde el INE.
    \item Tareas relacionadas con la gestión de los usuarios de la aplicación.
    \begin{itemize}
        \item Añadir la opción de editar el perfil del usuario.
        \item Encriptado de las contraseñas.
        \item Arreglar la confirmación de la creación de cuentas: Un usuario puede solicitar su registro en la aplicación para que posteriormente el superadministrador le acepte y su cuenta se active.
    \end{itemize}
\end{itemize}
\imagen{imagenes/sprint4}{Gráfica burndown del sprint 4}
\subsection{Quinto sprint: Introducción de datos desde el INE}
Este fue el sprint más costoso. Para su realización hicieron falta casi tres meses. En él las funcionalidades que se implementaron fueron:
\begin{itemize}
    \item Crear nuevas tablas en la base de datos para recoger los datos del INE y sus urls.
    \item Crear las vistas para mostrar los datos.
    \item Paso de datos desde el json proporcionado por el INE a la base de datos.
    \item Implementar funcionalidad para actualizar los datos de las variables del INE.  
    \item Crear la vista para indicar al usuario que se han actualizado los datos.
    \item Elegir la librería para exportar los datos de las tablas a excel.
    \item Arreglar la funcionalidad para exportar los datos a Excel.
\end{itemize}
\imagen{imagenes/sprint5}{Gráfica burndown del sprint 5}
\subsection{Sexto sprint: Mejora del tratamiento de datos}
En esta iteración se desarrollo la primera parte del análisis de los datos y su predicción a futuro.
Principales tareas:
\begin{itemize}
    \item Elegir la biblioteca para la predicción de datos.
    \item Pruebas con bibliotecas de machine learning como PHP-ML y Rubix. Al final escogí Rubix.
    \item Aprendizaje del uso de la biblioteca Rubix.
\end{itemize}
\imagen{imagenes/sprint6}{Gráfica burndown del sprint 6}
\subsection{Séptimo sprint: Configuración del entorno de pruebas}
Para que mis tutores pudieran probar mi código decidí usar Docker y Heroku.\\
Tanto la configuración de docker como la de Heroku me costaron mucho y me llevó más tiempo del esperado por lo que subí la duración de las tareas.\\
En este sprint también conecté codacy al repositorio de Github para la revisión del código.
\imagen{imagenes/sprint7}{Gráfica burndown del sprint 7}
\subsection{Octavo sprint: Configuración de tests y documentación}
Como no quería dejar la documentación para el final por si no me daba tiempo, comencé a escribirla antes de haber acabado el código.\\
En esta iteración además de escribir la memoria del proyecto (aunque será retocada en un sprint posterior), también se conectó los test de laravel al repositorio de Github para que se pasen de forma automática al hacer cambios en el mismo.\\
\imagen{imagenes/sprint8}{Gráfica burndown del sprint 8}
\subsection{Noveno sprint: Mejora de tratamiento de datos 2ª parte}
En este sprint las tareas fueron:
\begin{itemize}
    \item Implementación del análisis de datos y su predicción posterior.
    \item Creación de la vista de los datos predichos.
    \item Aspectos de configuración pendientes de Docker y Heroku.
\end{itemize}
\imagen{imagenes/sprint9}{Gráfica burndown del sprint 9}
\section{Estudio de viabilidad}
En esta sección se realizarán algunos cálculos para conocer los gastos que tendría el proyecto en una empresa real así como los temas legales que habría que solucionar.
\subsection{Viabilidad económica}

\subsection{Viabilidad legal}



\apendice{Especificación de Requisitos}

\section{Introducción}
En esta sección hablaré de los requisitos que cumple el proyecto en su totalidad aunque haré más hincapié en los que he creado yo.\\
Al no ser mi propio proyecto sino que he mejorado uno existente, muchos de los requisitos ya estaban implementados pero me ha parecido interesante mencionarlos y hablar de ellos para que el proyecto se entienda mejor. Además he mejorado o arreglado muchos de ellos.\\
Para poder diferenciar entre los que ya estaban en la aplicación y los que he creado yo los señalaré .\\
Es importante señalar que existen tres roles:
\begin{description}
    \item [Súper administrador] Tiene acceso a todas las funcionalidades del sistema. Se diferencia del administrador normal en que éste no tiene acceso a la gestión de los usuarios.
    \item [Administrador] Tiene acceso a todas las funcionalidades menos a la de gestión de los usuarios.
    \item [Invitado] Sólo se le permite acceder a la información de las tablas ya definidas y a la ayuda de la aplicación. Serán las personas que accedan a la aplicación a través del navegador.
\end{description}
\section{Objetivos generales}
Los objetivos generales del proyecto son:
\begin{itemize}
    \item Detectar y corregir los fallos del proyecto anterior.
    \item Permitir la entrada de datos del Instituto nacional de estadística (INE) de una forma más sencilla para el usuario.
    \item Mejorar la seguridad de la aplicación encriptando las contraseñas.
    \item Realizar una predicción de los datos tomando de referencia los existentes en la base de datos.
    \item Mejorar la ayuda de la aplicación para facilitar al usuario su utilización.
\end{itemize}
\section{Catalogo de requisitos}
A continuación, se enumeran los requisitos específicos derivados de los objetivos generales del proyecto.
\subsection{Requisitos funcionales}
\begin{description}
    \item [RF1 Registro en la aplicación web] Los nuevos usuarios podrán solicitar el registro en la aplicación por medio de un formulario con los siguientes campos: Nombre, correo electrónico y contraseña. Posteriormente el súper administrador aceptará la solicitud.
    \item [RF2 Login] Acceso a la aplicación.
    \item [RF3 Gestión de tablas] Gestión de las tablas de la aplicación.
    \begin{description}
        \item [RF3.1 Inserción de tablas] Insertar nuevas tablas a la base de datos.
        \item [RF3.2 Creación de tablas con categorías de cualquier variable] La aplicación podrá crear tablas con cualquier categoría de cualquier variable.
        \item [RF3.3 Modificación de tablas] Se permitirá modificar los valores que forman parte de una tabla.
        \begin{description}
            \item [RF3.3.1 Añadir año]
            \item [RF3.3.2 Añadir categoría] Se podrá añadir categorías nuevas y se podrá asignar la súper categoría a la que pertenecen.
            \item [RF3.3.3 Añadir ámbito geográfico] 
            \item [RF3.3.4 Borrar año]
            \item [RF3.3.5 Borrar categoría]
            \item [RF3.3.6 Borrar ámbito geográfico]
        \end{description}
        \item [RF3.4 Borrado de tablas]
    \end{description}
    \item [RF4 Creación de gráficos] Se generarán gráficos automáticamente a partir de los datos introducidos en la tabla, se podrá filtrar los datos que se desean ver en el gráfico por medio de un formulario.
    \item [RF5 Gestión de datos] Gestión de los datos de la aplicación.
    \begin{description}
        \item [RF5.1 Variables] Gestión de las variables.
        \begin{description}
            \item [RF5.1.1 Modificar] Se podrá modificar el nombre, la fuente, el tipo y la descripción.
            \item [RF5.1.2 Borrar] Se borrará una variable con todos los datos asociados a ella.
        \end{description}
        \item [RF5.2 Categorías] Gestión de las categorías.
        \begin{description}
            \item [RF5.2.1 Crear] Se creará una categoría y se la podrá asignar a una súper categoría, si no se selecciona ninguna, se introducirá a la súper categoría “Sin categoría”.
            \item [RF5.2.2 Modificar] Se podrá modificar el nombre y la súper categoría a la que está asignada una categoría.
            \item [RF5.2.3 Borrar]  Se podrá elegir si quieres borrar una categoría del sistema o tan solo de una variable.
        \end{description}
        \item [RF5.3 Súper categorías] Gestión de las súper categorías.
        \begin{description}
            \item [RF5.3.1 Crear] Se creará una súper categoría nueva y se le podrá asignar las categorías que están sin categoría.
            \item [RF5.3.2 Modificar] Se podrá modificar el nombre y las categorías que están asignadas.
            \item [RF5.3.3 Borrar] Borrar una súper categoría del sistema, si la súper categoría tenía categorías estas pasarán a “Sin categoría”.
        \end{description}
        \item [RF5.4 Ámbitos geográficos] Gestión de los ámbitos geográficos.
        \begin{description}
            \item [RF5.4.1 Crear] Se podrá crear un ámbito geográfico nuevo.
            \item [RF5.4.2 Modificar] Se modificará el nombre de un ámbito geográfico.
            \item [RF5.4.3 Borrar] Se podrá elegir si se desea borrar un ámbito del sistema o tan solo de una variable.
        \end{description}
    \end{description}
    \item [RF6 Exportación de tablas a Excel] La aplicación puede exportar las tablas creadas a partir de los datos almacenados en la base de datos a ficheros “.xls”.
    \item[RF7 Gestión de usuarios] Sólo tiene acceso a esta funcionalidad el súper administrador.
    \begin{description}
        \item[RF7.1 Modificar rol de los usuarios] Se podrá cambiar el rol de los usuarios de administrador a súper administrador.
        \item[RF7.2 Borrar usuario] Borrar usuario del sistema.
        \item[RF7.3 Gestión de peticiones de nuevos usuarios] Se podrá aceptar o declinar una petición de registro en la aplicación.
        \begin{description}
            \item[RF7.3.1 Aceptar petición] El usuario pasa a la base de datos como usuario de la aplicación.
            \item[RF7.3.2 Declinar petición] Se declina su solicitud y se borra de la tabla de usuarios por confirmar.
        \end{description}
    \end{description}
\end{description}
\section{Especificación de requisitos}



\apendice{Especificación de diseño}

\section{Introducción}

\section{Diseño de datos}

\section{Diseño procedimental}

\section{Diseño arquitectónico}



\apendice{Documentación técnica de programación}

\section{Introducción}

\section{Estructura de directorios}

\section{Manual del programador}

\section{Compilación, instalación y ejecución del proyecto}

\section{Pruebas del sistema}

\apendice{Documentación de usuario}

\section{Introducción}

\section{Requisitos de usuarios}

\section{Instalación}

\section{Manual del usuario}





\bibliographystyle{plain}
\bibliography{bibliografiaAnexos}

\end{document}
