\capitulo{1}{Introducción}

Todos los años, el departamento de economía de la Universidad de Burgos realiza un boletín exhaustivo sobre la coyuntura económica en el ámbito burgalés.\\
Para ello recopila datos de distintas fuentes: Instituto Nacional de estadística, banco nacional de España, Eurostat...\\
Este trabajo surge de la necesidad de una herramienta que permita organizar y tratar dichos datos, así como almacenarlos y facilitar el trabajo al departamento que los recoge.\\
El objetivo de la realización de los boletines de coyuntura económica es conocer el desarrollo de la economía del ámbito estudiado, mediante la producción de información económica y su divulgación a una amplio público de empresas, profesionales y particulares promoviendo así el análisis de la coyuntura económica.\\
Estos boletines, como ya he indicado antes, los crea el Equipo multidisciplinar de Coyuntura radicado en la Facultad de Ciencias Económicas y Empresariales de la Universidad de Burgos. En virtud del Convenio Marco de Colaboración firmado por la Universidad de Burgos y la actual Caja Viva Caja Rural.\\
Este equipo multidisciplinar integrado por 16 profesores, de los Departamentos de Economía Aplicada, Economía y Administración de Empresas y Derecho analizan la evolución económica coyuntural de la provincia de Burgos.\\
La aplicación va dirigida al equipo de coyuntura económica, por lo que la mayoría de los usuarios que van a manejar esta aplicación van a ser administradores, debido a esto, se ha creado el rol del super administrador, para tener un control de los administradores ya que estos no son usuarios experimentados.\\
Por lo tanto, la aplicación proporciona diferentes funcionalidades que dependerán del rol (super administrador, administrador y invitado) del usuario que haya accedido.\\
La principal funcionalidad de la aplicación consiste en la introducción de datos estadísticos de la coyuntura económica dependientes de una categoría específica, un ámbito geográfico y un
año.\\
A partir de estos datos se generarán tablas que podrán ser filtradas y mostradas al usuario de la manera que a éste le resulte mas cómoda y utilizarlas para realizar el boletín.\\
La aplicación también genera gráficos que muestran los datos de una forma sencilla e intuitiva.\\
Los invitados que accedan a la aplicación deberán tener un rol solo de lectura, por lo que sólo podrán filtrar y visualizar las tablas y gráficos de las variables económicas que están almacenadas en la base de datos de la aplicación.\\
Los encargados de introducir datos estadísticos, modificarlos o borrar los que estén obsoletos o erróneos, serán los administradores de la aplicación.\\
La aplicación cuenta con una ayuda sobre cómo manejar las funcionalidades de la aplicación, esta ayuda varía dependiendo del rol del usuario conectado.\\
Mi trabajo consiste en mejorar la aplicación, corregir algunos errores y añadir nuevas funcionalidades como la de introducir datos de forma automatizada desde el Instituto Nacional de Estadística (INE).\\


