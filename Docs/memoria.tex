\documentclass[a4paper,13pt,twoside]{memoir}

% Castellano
\usepackage[spanish,es-tabla]{babel}
\selectlanguage{spanish}
\usepackage[utf8]{inputenc}
\usepackage[T1]{fontenc}
\usepackage{lmodern} % Scalable font
\usepackage{microtype}
\usepackage{placeins}

\RequirePackage{booktabs}
\RequirePackage[table]{xcolor}
\RequirePackage{xtab}
\RequirePackage{multirow}

% Links
\usepackage[colorlinks]{hyperref}
\hypersetup{
	allcolors = {red}
}

% Ecuaciones
\usepackage{amsmath}

% Rutas de fichero / paquete
\newcommand{\ruta}[1]{{\sffamily #1}}

% Párrafos
\nonzeroparskip


% Imagenes
\usepackage{graphicx}
\newcommand{\imagen}[2]{
	\begin{figure}[!h]
		\centering
		\includegraphics[width=0.9\textwidth]{#1}
		\caption{#2}\label{fig:#1}
	\end{figure}
	\FloatBarrier
}

\newcommand{\imagenflotante}[2]{
	\begin{figure}%[!h]
		\centering
		\includegraphics[width=0.9\textwidth]{#1}
		\caption{#2}\label{fig:#1}
	\end{figure}
}



% El comando \figura nos permite insertar figuras comodamente, y utilizando
% siempre el mismo formato. Los parametros son:
% 1 -> Porcentaje del ancho de página que ocupará la figura (de 0 a 1)
% 2 --> Fichero de la imagen
% 3 --> Texto a pie de imagen
% 4 --> Etiqueta (label) para referencias
% 5 --> Opciones que queramos pasarle al \includegraphics
% 6 --> Opciones de posicionamiento a pasarle a \begin{figure}
\newcommand{\figuraConPosicion}[6]{%
  \setlength{\anchoFloat}{#1\textwidth}%
  \addtolength{\anchoFloat}{-4\fboxsep}%
  \setlength{\anchoFigura}{\anchoFloat}%
  \begin{figure}[#6]
    \begin{center}%
      \Ovalbox{%
        \begin{minipage}{\anchoFloat}%
          \begin{center}%
            \includegraphics[width=\anchoFigura,#5]{#2}%
            \caption{#3}%
            \label{#4}%
          \end{center}%
        \end{minipage}
      }%
    \end{center}%
  \end{figure}%
}

%
% Comando para incluir imágenes en formato apaisado (sin marco).
\newcommand{\figuraApaisadaSinMarco}[5]{%
  \begin{figure}%
    \begin{center}%
    \includegraphics[angle=90,height=#1\textheight,#5]{#2}%
    \caption{#3}%
    \label{#4}%
    \end{center}%
  \end{figure}%
}
% Para las tablas
\newcommand{\otoprule}{\midrule [\heavyrulewidth]}
%
% Nuevo comando para tablas pequeñas (menos de una página).
\newcommand{\tablaSmall}[5]{%
 \begin{table}
  \begin{center}
   \rowcolors {2}{gray!35}{}
   \begin{tabular}{#2}
    \toprule
    #4
    \otoprule
    #5
    \bottomrule
   \end{tabular}
   \caption{#1}
   \label{tabla:#3}
  \end{center}
 \end{table}
}

%
% Nuevo comando para tablas pequeñas (menos de una página).
\newcommand{\tablaSmallSinColores}[5]{%
 \begin{table}[H]
  \begin{center}
   \begin{tabular}{#2}
    \toprule
    #4
    \otoprule
    #5
    \bottomrule
   \end{tabular}
   \caption{#1}
   \label{tabla:#3}
  \end{center}
 \end{table}
}

\newcommand{\tablaApaisadaSmall}[5]{%
\begin{landscape}
  \begin{table}
   \begin{center}
    \rowcolors {2}{gray!35}{}
    \begin{tabular}{#2}
     \toprule
     #4
     \otoprule
     #5
     \bottomrule
    \end{tabular}
    \caption{#1}
    \label{tabla:#3}
   \end{center}
  \end{table}
\end{landscape}
}

%
% Nuevo comando para tablas grandes con cabecera y filas alternas coloreadas en gris.
\newcommand{\tabla}[6]{%
  \begin{center}
    \tablefirsthead{
      \toprule
      #5
      \otoprule
    }
    \tablehead{
      \multicolumn{#3}{l}{\small\sl continúa desde la página anterior}\\
      \toprule
      #5
      \otoprule
    }
    \tabletail{
      \hline
      \multicolumn{#3}{r}{\small\sl continúa en la página siguiente}\\
    }
    \tablelasttail{
      \hline
    }
    \bottomcaption{#1}
    \rowcolors {2}{gray!35}{}
    \begin{xtabular}{#2}
      #6
      \bottomrule
    \end{xtabular}
    \label{tabla:#4}
  \end{center}
}

%
% Nuevo comando para tablas grandes con cabecera.
\newcommand{\tablaSinColores}[6]{%
  \begin{center}
    \tablefirsthead{
      \toprule
      #5
      \otoprule
    }
    \tablehead{
      \multicolumn{#3}{l}{\small\sl continúa desde la página anterior}\\
      \toprule
      #5
      \otoprule
    }
    \tabletail{
      \hline
      \multicolumn{#3}{r}{\small\sl continúa en la página siguiente}\\
    }
    \tablelasttail{
      \hline
    }
    \bottomcaption{#1}
    \begin{xtabular}{#2}
      #6
      \bottomrule
    \end{xtabular}
    \label{tabla:#4}
  \end{center}
}

%
% Nuevo comando para tablas grandes sin cabecera.
\newcommand{\tablaSinCabecera}[5]{%
  \begin{center}
    \tablefirsthead{
      \toprule
    }
    \tablehead{
      \multicolumn{#3}{l}{\small\sl continúa desde la página anterior}\\
      \hline
    }
    \tabletail{
      \hline
      \multicolumn{#3}{r}{\small\sl continúa en la página siguiente}\\
    }
    \tablelasttail{
      \hline
    }
    \bottomcaption{#1}
  \begin{xtabular}{#2}
    #5
   \bottomrule
  \end{xtabular}
  \label{tabla:#4}
  \end{center}
}



\definecolor{cgoLight}{HTML}{EEEEEE}
\definecolor{cgoExtralight}{HTML}{FFFFFF}

%
% Nuevo comando para tablas grandes sin cabecera.
\newcommand{\tablaSinCabeceraConBandas}[5]{%
  \begin{center}
    \tablefirsthead{
      \toprule
    }
    \tablehead{
      \multicolumn{#3}{l}{\small\sl continúa desde la página anterior}\\
      \hline
    }
    \tabletail{
      \hline
      \multicolumn{#3}{r}{\small\sl continúa en la página siguiente}\\
    }
    \tablelasttail{
      \hline
    }
    \bottomcaption{#1}
    \rowcolors[]{1}{cgoExtralight}{cgoLight}

  \begin{xtabular}{#2}
    #5
   \bottomrule
  \end{xtabular}
  \label{tabla:#4}
  \end{center}
}


















\graphicspath{ {./img/} }

% Capítulos
\chapterstyle{bianchi}
\newcommand{\capitulo}[2]{
	\setcounter{chapter}{#1}
	\setcounter{section}{0}
	\chapter*{#2}
	\addcontentsline{toc}{chapter}{#2}
	\markboth{#2}{#2}
}

% Apéndices
\renewcommand{\appendixname}{Apéndice}
\renewcommand*\cftappendixname{\appendixname}

\newcommand{\apendice}[1]{
	%\renewcommand{\thechapter}{A}
	\chapter{#1}
}

\renewcommand*\cftappendixname{\appendixname\ }

% Formato de portada
\makeatletter
\usepackage{xcolor}
\newcommand{\tutor}[1]{\def\@tutor{#1}}
\newcommand{\course}[1]{\def\@course{#1}}
\definecolor{cpardoBox}{HTML}{E6E6FF}
\def\maketitle{
  \null
  \thispagestyle{empty}
  % Cabecera ----------------
\noindent\includegraphics[width=\textwidth]{cabecera}\vspace{1cm}%
  \vfill
  % Título proyecto y escudo informática ----------------
  \colorbox{cpardoBox}{%
    \begin{minipage}{.8\textwidth}
      \vspace{.5cm}\Large
      \begin{center}
      \textbf{TFG del Grado en Ingeniería Informática}\vspace{.6cm}\\
      \textbf{\LARGE\@title{}}
      \end{center}
      \vspace{.2cm}
    \end{minipage}

  }%
  \hfill\begin{minipage}{.20\textwidth}
    \includegraphics[width=\textwidth]{escudoInfor}
  \end{minipage}
  \vfill
  % Datos de alumno, curso y tutores ------------------
  \begin{center}%
  {%
    \noindent\LARGE
    Presentado por \@author{}\\ 
    en Universidad de Burgos --- \@date{}\\
    Tutor: \@tutor{}\\
  }%
  \end{center}%
  \null
  \cleardoublepage
  }
\makeatother

\newcommand{\nombre}{Raquel Sancha Sánchez} %%% cambio de comando

% Datos de portada
\title{título del TFG}
\author{\nombre}
\tutor{nombre tutor}
\date{\today}

\begin{document}

\maketitle


\newpage\null\thispagestyle{empty}\newpage


%%%%%%%%%%%%%%%%%%%%%%%%%%%%%%%%%%%%%%%%%%%%%%%%%%%%%%%%%%%%%%%%%%%%%%%%%%%%%%%%%%%%%%%%
\thispagestyle{empty}


\noindent\includegraphics[width=\textwidth]{cabecera}\vspace{1cm}

\noindent D. nombre tutor, profesor del departamento de nombre departamento, área de nombre área.

\noindent Expone:

\noindent Que el alumno D. \nombre, con DNI dni, ha realizado el Trabajo final de Grado en Ingeniería Informática titulado título de TFG. 

\noindent Y que dicho trabajo ha sido realizado por el alumno bajo la dirección del que suscribe, en virtud de lo cual se autoriza su presentación y defensa.

\begin{center} %\large
En Burgos, {\large \today}
\end{center}

\vfill\vfill\vfill

% Author and supervisor
\begin{minipage}{0.45\textwidth}
\begin{flushleft} %\large
Vº. Bº. del Tutor:\\[2cm]
D. nombre tutor
\end{flushleft}
\end{minipage}
\hfill
\begin{minipage}{0.45\textwidth}
\begin{flushleft} %\large
Vº. Bº. del co-tutor:\\[2cm]
D. nombre co-tutor
\end{flushleft}
\end{minipage}
\hfill

\vfill

% para casos con solo un tutor comentar lo anterior
% y descomentar lo siguiente
%Vº. Bº. del Tutor:\\[2cm]
%D. nombre tutor


\newpage\null\thispagestyle{empty}\newpage




\frontmatter

% Abstract en castellano
\renewcommand*\abstractname{Resumen}
\begin{abstract}
En este primer apartado se hace una \textbf{breve} presentación del tema que se aborda en el proyecto.
\end{abstract}

\renewcommand*\abstractname{Descriptores}
\begin{abstract}
Palabras separadas por comas que identifiquen el contenido del proyecto Ej: servidor web, buscador de vuelos, android \ldots
\end{abstract}

\clearpage

% Abstract en inglés
\renewcommand*\abstractname{Abstract}
\begin{abstract}
A \textbf{brief} presentation of the topic addressed in the project.
\end{abstract}

\renewcommand*\abstractname{Keywords}
\begin{abstract}
keywords separated by commas.
\end{abstract}

\clearpage

% Indices
\tableofcontents

\clearpage

\listoffigures

\clearpage

\listoftables
\clearpage

\mainmatter
\capitulo{1}{Introducción}
Todos los años, el Departamento de Economía Aplicada de la Universidad de Burgos realiza un boletín exhaustivo sobre la coyuntura económica en el ámbito burgalés (\href{https://www.ubu.es/departamento-de-economia-aplicada/investigacion-research/grupos-de-investigacion-research-groups/equipo-de-coyuntura-economica-de-burgos}{Enlace a algunos boletines}).\\
Estos boletines los crea el Equipo multidisciplinar de Coyuntura radicado en la Facultad de Ciencias Económicas y Empresariales de la Universidad de Burgos. En virtud del Convenio Marco de Colaboración firmado por la Universidad de Burgos y la actual Caja Viva Caja Rural.\\
Este equipo multidisciplinar integrado por 16 profesores, de los Departamentos de Economía Aplicada, Economía y Administración de Empresas y Derecho analiza la evolución económica coyuntural de la provincia de Burgos.\\
Para ello recopila datos de distintas fuentes: Instituto Nacional de estadística (INE), Banco Nacional de España, Eurostat...\\
El objetivo de la realización de los boletines de coyuntura económica es conocer el desarrollo de la economía del ámbito estudiado, mediante el análisis de información económica y su divulgación a un amplio público de empresas, profesionales y particulares.\\
Aunque el cliente final de la aplicación es este equipo, también se permitirá el acceso a invitados para que puedan ver la información recogida por lo que existe un sistema de roles con distintos permisos que se explicarán más adelante.\\
La aplicación se creó en 2017 para usarse como una herramienta que permita organizar y tratar dichos datos, así como almacenarlos y facilitar el trabajo al departamento que los recoge.\\
La aplicación proporciona una interfaz web que permite la entrada y el almacenamiento de variables e incluye herramientas para la visualización de datos entre otras funcionalidades.\\
A partir de estos datos se generan tablas que podrán ser filtradas y mostradas al usuario de la manera que a éste le resulte mas cómoda y utilizarlas para realizar el boletín.\\
La aplicación también genera gráficos que muestran los datos de una forma sencilla e intuitiva.\\
Los invitados que accedan a la aplicación deberán tener un rol solo de lectura, por lo que sólo podrán filtrar y visualizar las tablas y gráficos de las variables económicas que están almacenadas en la base de datos de la aplicación.\\
Los encargados de introducir datos estadísticos, modificarlos o borrar los que estén obsoletos o erróneos, serán los administradores de la aplicación.\\
Este trabajo consiste en mejorar dicha aplicación y realizar un proceso como es el mantenimiento de software en un entorno profesional.\\ 
La aportación que hace este trabajo a la aplicación se basa en realizar algunas mejoras, corregir errores y añadir nuevas funcionalidades como la de introducir datos de forma automatizada desde el INE o la predicción de datos.\\



\capitulo{2}{Objetivos del proyecto}
\section{Objetivos generales}
\begin{itemize}
	\item Arreglar los errores existentes en la aplicación como por ejemplo los problemas en la administración de usuarios. 
	\item Ampliar la funcionalidad de la aplicación permitiendo introducir datos desde el INE.
	\item Permitir exportar los datos de las tablas a Excel.
	\item Permitir actualizar los datos sacados del INE.
	\item Mejorar la ayuda a los usuarios.
	\item Subir la aplicación al servidor de la UBU.
\end{itemize}
\section{Objetivos técnicos}
\begin{itemize}
	\item Usar la arquitectura de laravel que consiste en modelos, vistas y controladores.
	\item Hacer uso de visual studio code para el desarrollo del código.
	\item Usar el servidor xampp con la base de datos de mysql.
	\item Conseguir transferir los datos del INE en formato JSON a las tablas de mi base de datos.
	\item Exportar correctamente los datos de las tablas a Excel.
	\item Mejorar la administración de usuarios mediante el uso de la plantilla entrust-gui.
	\item Trabajar con la base de datos MySql.
	\item Utilizar github para el seguimiento de mi trabajo.
	\item Usar \LaTeX para la documentación del proyecto
\end{itemize}
\section{Objetivos personales}
\begin{itemize}
	\item Aprender a programar en php.
	\item Aprender la metodología de laravel y de las aplicaciones web.
	\item Entender el código escrito por otra persona y modificarlo a mi gusto.
\end{itemize}
\capitulo{3}{Conceptos teóricos}
 \section{Procesado de datos del INE}
El instituto nacional de estadística proporciona un servicio que permite descargar los datos que nos ofrece en formato json. Para ello debemos crear una petición en forma de url para después guardar esos datos e introducirlos en nuestra base de datos.
 \subsection{Creación de la url}
 Para comprender este paso hace falta saber cómo es la estructura de una petición url para el servicio de datos abiertos del INE. \cite{ine:urljson}\\
 \imagen{imagenes/urlJSON}{Estructura de la url}
Los campos que aparecen entre llaves, \{ \}, son obligatorios.\\
Los campos que aparecen entre corchetes, [ ], son opcionales y cambian en relación a la función considerada. En la construcción de nuestras url no utilizo estos campos.\\
Descripción de cada uno de ellos
\begin{description}
	\item [idioma] ES para español e EN para inglés. Por defecto está puesto el español.\\
	\item [función] Funciones implementadas en el sistema para poder realizar diferentes tipos de consulta en función del tipo de fuente, Tempus3 o PcAxis, y del elemento que se quiere obtener.\\
    Funciones para la obtención de datos de Tempus3
         \begin{description}
         \item [Operaciones] OPERACIONES\_DISPONIBLES, OPERACIÓN.
         \item [Variables] VARIABLES, VARIABLES\_OPERACION.
         \item [Valores] VALORES\_VARIABLES, VALORES\_VARIABLEOPERACION.
         \item [Tablas] TABLAS\_OPERACION, GRUPOS\_TABLA.
         \item [Series] SERIE, SERIES\_OPERACION.
         \item [Publicaciones] PUBLICACIONES, PUBLICACIONES\_OPERACION.
         \item [Datos] DATOS\_SERIE, DATOS\_TABLA.
         \end{description}
    Función para la obtención de datos del repositorio de ficheros PcAxis Al ser Pc-Axis un formato para difundir tablas estadísticas, la única función implementada es la siguiente:
         \begin{description}
         \item [Datos] DATOS\_TABLA
         \end{description}
    \item [inputs] Identificadores de los elementos de entrada de las funciones. Estos inputs varían en base a la función utilizada.
    Existen dos tipos de repositorios de los que el INE saca sus datos; los repositorios de tablas Tempus3 y los de PC-Axis.
    \imagen{imagenes/idTempus3}{Identificador de las tablas Tempus3}
    \imagen{imagenes/idPcAxis}{Identificador de las tablas PC-Axis}
    \item [parámetros] Los parámetros en la URL se establecen a partir del símbolo ?.\\
    Cuando haya más de un parámetro, el símbolo & se utiliza como separador.\\
    No todas las funciones admiten todos los parámetros posibles. Por ello haremos una clasificación para explicarlos mejor:  
     \begin{enumerate}
        \item Parámetros comunes a todas las funciones
            \begin{description}
            \item [page] Si hay más de 500 elementos, la consulta se divide en páginas. Esta opción nos permite seleccionar la página que queremos visualizar.
            \item [download] Para descargarnos el fichero json.
            \item [det] Este parámetro da más detalles de la información mostrada.
            \item [tip] Cambia la forma de mostrar la información.
            \end{description}
        \item Parámetros para la petición de datos
            \begin{description}
                \item [date] Filtra los datos por fecha; fecha concreta, lista o rango de fechas.
                \item [nult] Devuelve los últimos n datos. Ejemplo: nult=4 devuelve los 4 últimos datos.
            \end{description}
        \item Parámetros para la obtención de datos y metadatos en base al ámbito geográfico
            \begin{description}
                \item [geo] Con geo = 1 para provincias, municipios u otras desagregaciones y geo = 0 para datos nacionales.
            \end{description}
        \end{enumerate}
\end{description}
 Para facilitar el trabajo a los usuarios de la aplicación, la petición url se genera automáticamente a partir de la url de la página del INE donde se encuentren los datos que queremos.
 \imagen{imagenes/Ine}{Ejemplo de una tabla del INE}
 \imagen{imagenes/urlINE}{Ejemplo de la url del INE}
 El usuario copia y pega esta url y la aplicación la traduce a una petición de los datos en forma de json automáticamente.
 \subsection{Estructura de los datos del INE}
 Para entender cómo he tratado los datos, necesitamos saber primero su significado.




Algunos conceptos teóricos de \LaTeX \footnote{Créditos a los proyectos de Álvaro López Cantero: Configurador de Presupuestos y Roberto Izquierdo Amo: PLQuiz}.

\section{Secciones}

Las secciones se incluyen con el comando section.

\subsection{Subsecciones}

Además de secciones tenemos subsecciones.

\subsubsection{Subsubsecciones}

Y subsecciones. 


\section{Referencias}

Las referencias se incluyen en el texto usando cite \cite{wiki:latex}. Para citar webs, artículos o libros \cite{koza92}.


\section{Imágenes}

Se pueden incluir imágenes con los comandos standard de \LaTeX, pero esta plantilla dispone de comandos propios como por ejemplo el siguiente:

\imagen{escudoInfor}{Autómata para una expresión vacía}



\section{Listas de items}

Existen tres posibilidades:

\begin{itemize}
	\item primer item.
	\item segundo item.
\end{itemize}

\begin{enumerate}
	\item primer item.
	\item segundo item.
\end{enumerate}

\begin{description}
	\item[Primer item] más información sobre el primer item.
	\item[Segundo item] más información sobre el segundo item.
\end{description}
	
\begin{itemize}
\item 
\end{itemize}

\section{Tablas}

Igualmente se pueden usar los comandos específicos de \LaTeX o bien usar alguno de los comandos de la plantilla.

\tablaSmall{Herramientas y tecnologías utilizadas en cada parte del proyecto}{l c c c c}{herramientasportipodeuso}
{ \multicolumn{1}{l}{Herramientas} & App AngularJS & API REST & BD & Memoria \\}{ 
HTML5 & X & & &\\
CSS3 & X & & &\\
BOOTSTRAP & X & & &\\
JavaScript & X & & &\\
AngularJS & X & & &\\
Bower & X & & &\\
PHP & & X & &\\
Karma + Jasmine & X & & &\\
Slim framework & & X & &\\
Idiorm & & X & &\\
Composer & & X & &\\
JSON & X & X & &\\
PhpStorm & X & X & &\\
MySQL & & & X &\\
PhpMyAdmin & & & X &\\
Git + BitBucket & X & X & X & X\\
Mik\TeX{} & & & & X\\
\TeX{}Maker & & & & X\\
Astah & & & & X\\
Balsamiq Mockups & X & & &\\
VersionOne & X & X & X & X\\
} 

\capitulo{4}{Técnicas y herramientas}

Esta parte de la memoria tiene como objetivo presentar las técnicas metodológicas y las herramientas de desarrollo que se han utilizado para llevar a cabo el proyecto. Si se han estudiado diferentes alternativas de metodologías, herramientas, bibliotecas se puede hacer un resumen de los aspectos más destacados de cada alternativa, incluyendo comparativas entre las distintas opciones y una justificación de las elecciones realizadas. 
No se pretende que este apartado se convierta en un capítulo de un libro dedicado a cada una de las alternativas, sino comentar los aspectos más destacados de cada opción, con un repaso somero a los fundamentos esenciales y referencias bibliográficas para que el lector pueda ampliar su conocimiento sobre el tema.



\capitulo{5}{Aspectos relevantes del desarrollo del proyecto}

Este apartado pretende recoger los aspectos más interesantes del desarrollo del proyecto, comentados por los autores del mismo.
Debe incluir desde la exposición del ciclo de vida utilizado, hasta los detalles de mayor relevancia de las fases de análisis, diseño e implementación.
Se busca que no sea una mera operación de copiar y pegar diagramas y extractos del código fuente, sino que realmente se justifiquen los caminos de solución que se han tomado, especialmente aquellos que no sean triviales.
Puede ser el lugar más adecuado para documentar los aspectos más interesantes del diseño y de la implementación, con un mayor hincapié en aspectos tales como el tipo de arquitectura elegido, los índices de las tablas de la base de datos, normalización y desnormalización, distribución en ficheros3, reglas de negocio dentro de las bases de datos (EDVHV GH GDWRV DFWLYDV), aspectos de desarrollo relacionados con el WWW...
Este apartado, debe convertirse en el resumen de la experiencia práctica del proyecto, y por sí mismo justifica que la memoria se convierta en un documento útil, fuente de referencia para los autores, los tutores y futuros alumnos.

\capitulo{6}{Trabajos relacionados}

Este apartado sería parecido a un estado del arte de una tesis o tesina. En un trabajo final grado no parece obligada su presencia, aunque se puede dejar a juicio del tutor el incluir un pequeño resumen comentado de los trabajos y proyectos ya realizados en el campo del proyecto en curso. 

\capitulo{7}{Conclusiones y Líneas de trabajo futuras}

Todo proyecto debe incluir las conclusiones que se derivan de su desarrollo. Éstas pueden ser de diferente índole, dependiendo de la tipología del proyecto, pero normalmente van a estar presentes un conjunto de conclusiones relacionadas con los resultados del proyecto y un conjunto de conclusiones técnicas. 
Además, resulta muy útil realizar un informe crítico indicando cómo se puede mejorar el proyecto, o cómo se puede continuar trabajando en la línea del proyecto realizado. 



\bibliographystyle{plain}
\bibliography{bibliografia}

\end{document}
