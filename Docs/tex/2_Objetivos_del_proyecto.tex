\capitulo{2}{Objetivos del proyecto}
\section{Objetivos generales}
\begin{itemize}
	\item Identificar los errores existentes en la aplicación. 
	\item Corregir los errores. 
	\item Ampliar la funcionalidad de la aplicación permitiendo introducir datos desde el INE.
	\item Realizar predicciones de los datos y mostrarlas en gráficos.
	\item Permitir exportar los datos de las tablas a Excel.
	\item Mejorar la ayuda a los usuarios.
	\item Implementar la edición de usuarios.
	\item Mejorar la seguridad de la aplicación.
	\item Publicar la aplicación en su entorno de explotación final que es el servidor de la UBU.
	\item Crear una aplicación web funcional y operativa que funcione sin errores.
\end{itemize}
\section{Objetivos técnicos}
\begin{itemize}
	\item Aprovechar las ventajas de la arquitectura de Laravel que consiste en modelos, vistas y controladores.
	\item Hacer uso de visual studio code para el desarrollo del código.
	\item Conseguir transferir los datos del INE en formato JSON a las tablas de la base de datos.
	\item Exportar correctamente los datos de las tablas a Excel.
	\item Encriptar las contraseñas usando un algoritmo seguro y eficiente.
	\item Mejorar la administración de usuarios.
\end{itemize}
\section{Objetivos personales}
\begin{itemize}
	\item Aprender a programar en PHP.
	\item Aprender la metodología de Laravel y de las aplicaciones web.
	\item Entender el código escrito por otra persona y modificarlo según las necesidades para realizar procedimientos comunes en el desarrollo de software profesional, como el mantenimiento del software, refactorización de código, etc.
	
\end{itemize}