\apendice{Especificación de Requisitos}

\section{Introducción}
En esta sección se hablará de los requisitos que cumple la aplicación en su totalidad aunque se remarcará los que se han conseguido con este trabajo.\\
Al no ser un proyecto que se haya empezado desde cero, sino que se ha mejorado uno existente, muchos de los requisitos ya estaban implementados pero es interesante mencionarlos y hablar de ellos para que la funcionalidad de la aplicacion se entienda mejor. Además se ha mejorado o arreglado muchos de ellos.\\
Es importante señalar que existen tres roles:
\begin{description}
    \item [Súper administrador] Tiene acceso a todas las funcionalidades del sistema. Se diferencia del administrador normal en que este último no tiene acceso a la gestión de los usuarios.
    \item [Administrador] Tiene acceso a todas las funcionalidades menos a la de gestión de los usuarios.
    \item [Invitado] Sólo se le permite acceder a la información de las tablas ya definidas y a la ayuda de la aplicación. Serán las personas que accedan a la aplicación a través del navegador.
\end{description}
\section{Objetivos generales}
Los objetivos generales del proyecto son:
\begin{itemize}
    \item Detectar y corregir los fallos del proyecto anterior.
    \item Permitir la entrada de datos del Instituto nacional de estadística (INE) de una forma más sencilla para el usuario.
    \item Mejorar la seguridad de la aplicación encriptando las contraseñas.
    \item Realizar una predicción de los datos tomando de referencia los existentes en la base de datos.
    \item Mejorar la ayuda de la aplicación para facilitar al usuario su utilización.
\end{itemize}
\section{Catalogo de requisitos}
A continuación, se enumeran los requisitos específicos derivados de los objetivos generales del proyecto.
\subsection{Requisitos funcionales}
\subsection{Requisitos originales}
\begin{description}
    \item [RF1 Registro en la aplicación web] Los nuevos usuarios podrán solicitar el registro en la aplicación por medio de un formulario con los siguientes campos: Nombre, correo electrónico y contraseña. Posteriormente el súper administrador aceptará la solicitud.
    \item [RF2 Login] Acceso a la aplicación.
    \item [RF3 Gestión de tablas] Gestión de las tablas de la aplicación.
    \begin{description}
        \item [RF3.1 Inserción de tablas] Insertar nuevas tablas a la base de datos.
        \item [RF3.2 Creación de tablas con categorías de cualquier variable] La aplicación podrá crear tablas con cualquier categoría de cualquier variable.
        \item [RF3.3 Modificación de tablas] Se permitirá modificar los valores que forman parte de una tabla.
        \begin{description}
            \item [RF3.3.1 Añadir año]
            \item [RF3.3.2 Añadir categoría] Se podrá añadir categorías nuevas y se podrá asignar la súper categoría a la que pertenecen.
            \item [RF3.3.3 Añadir ámbito geográfico] 
            \item [RF3.3.4 Borrar año]
            \item [RF3.3.5 Borrar categoría]
            \item [RF3.3.6 Borrar ámbito geográfico]
        \end{description}
        \item [RF3.4 Visualización de tablas]
        \item [RF3.5 Borrado de tablas]
    \end{description}
    \item [RF4 Creación de gráficos] Se generarán gráficos automáticamente a partir de los datos introducidos en la tabla, se podrá filtrar los datos que se desean ver en el gráfico por medio de un formulario.
    \item [RF5 Gestión de datos] Gestión de los datos de la aplicación.
    \begin{description}
        \item [RF5.1 Variables] Gestión de las variables.
        \begin{description}
            \item [RF5.1.1 Modificar] Se podrá modificar el nombre, la fuente, el tipo y la descripción.
            \item [RF5.1.2 Borrar] Se borrará una variable con todos los datos asociados a ella.
        \end{description}
        \item [RF5.2 Categorías] Gestión de las categorías.
        \begin{description}
            \item [RF5.2.1 Crear] Se creará una categoría y se la podrá asignar a una súper categoría, si no se selecciona ninguna, se introducirá a la súper categoría “Sin categoría”.
            \item [RF5.2.2 Modificar] Se podrá modificar el nombre y la súper categoría a la que está asignada una categoría.
            \item [RF5.2.3 Borrar]  Se podrá elegir si quieres borrar una categoría del sistema o tan solo de una variable.
        \end{description}
        \item [RF5.3 Súper categorías] Gestión de las súper categorías.
        \begin{description}
            \item [RF5.3.1 Crear] Se creará una súper categoría nueva y se le podrá asignar las categorías que están sin categoría.
            \item [RF5.3.2 Modificar] Se podrá modificar el nombre y las categorías que están asignadas.
            \item [RF5.3.3 Borrar] Borrar una súper categoría del sistema, si la súper categoría tenía categorías estas pasarán a “Sin categoría”.
        \end{description}
        \item [RF5.4 Ámbitos geográficos] Gestión de los ámbitos geográficos.
        \begin{description}
            \item [RF5.4.1 Crear] Se podrá crear un ámbito geográfico nuevo.
            \item [RF5.4.2 Modificar] Se modificará el nombre de un ámbito geográfico.
            \item [RF5.4.3 Borrar] Se podrá elegir si se desea borrar un ámbito del sistema o tan solo de una variable.
        \end{description}
    \end{description}
    \subsection{Requisitos nuevos}
    \item [RF6 Exportación de tablas a Excel] La aplicación puede exportar las tablas creadas a partir de los datos almacenados en la base de datos a ficheros \textit{.xls}.
    \item[RF7 Gestión de usuarios] Sólo tiene acceso a esta funcionalidad el súper administrador.
    \begin{description}
        \item[RF7.1 Modificar rol de los usuarios] Se podrá cambiar el rol de los usuarios de administrador a súper administrador.
        \item[RF7.2 Borrar usuario] Borrar usuario del sistema.
        \item[RF7.3 Gestión de peticiones de nuevos usuarios] Se podrá aceptar o declinar una petición de registro en la aplicación.
        \begin{description}
            \item[RF7.3.1 Aceptar petición] El usuario pasa a la base de datos como usuario de la aplicación.
            \item[RF7.3.2 Declinar petición] Se declina su solicitud y se borra de la tabla de usuarios por confirmar.
        \end{description}
    \end{description}
    \item[RF8 Gestión de datos desde el INE] Se permite la introducción y actualización de datos del Instituto Nacional de Estadística a las tablas de la aplicación.
    \begin{description}
        \item[RF8.1 Introducción de datos] El administrador o el súper administrador pueden introducir datos a las tablas de la aplicación a partir de la dirección web de una página del INE.
        \item[RF8.2 Actualización de datos] El administrador o el súper administrador pueden actualizar las tablas que contengan datos que provengan del INE con información nueva.
    \end{description}
    \item[RF9 Predicción de datos] Todos los usuarios pueden acceder  a una predicción del siguiente año de las variables de la aplicación.
    \item[RF10 Editar perfil del usuario] Cada usuario podrá editar los datos de su perfil.
    \item[RF11 Ayuda de la aplicación] El usuario debe poder obtener ayuda sobre el uso de las funcionalidades de la aplicación.
\end{description}
\subsection{Requisitos no funcionales}
\begin{description}
    \item[RNF1 Usabilidad] La aplicación debe ser intuitiva, con una curva baja de aprendizaje y adaptada al entorno de trabajo.
    \item[RNF2 Rendimiento] La aplicación debe tener unos tiempos de carga razonables y aceptables.
    \item[RNF3 Escalabilidad] Debe permitir la adición de nuevas funcionalidades sin problemas.
    \item[RNF4 Disponibilidad] Debe estar disponible y operativa en cualquier momento.
    \item[RNF5 Seguridad] Debe gestionar la información sensible, como las contraseñas de una forma correcta.
\end{description}
\subsection{Requisitos de restricción}
\begin{description}
    \item[RR1]  La aplicación prohíbe a los invitados la modificación o borrado de los datos introducidos, por lo que solo podrán visualizar las tablas y gráficos creadas por los administradores de la aplicación web.
    \item[RR2] Los administradores/editores no podrán tener acceso a la gestión de usuarios, por lo que no podrán modificar datos de los usuarios registrados ni administrar las peticiones de registro.
\end{description}
\section{Especificación de requisitos}
En esta sección se mostrará los diagramas de casos de uso resultante y se desarrollará cada uno de ellos.
\subsection{Diagramas de casos de uso}
\subsection{Actores}
\begin{description}
    \item [Súper administrador] Tiene acceso a todas las funcionalidades del sistema. Se diferencia del administrador normal en que este último no tiene acceso a la gestión de los usuarios.
    \item [Administrador] Tiene acceso a todas las funcionalidades menos a la de gestión de los usuarios.
    \item [Invitado] Sólo se le permite acceder a la información de las tablas ya definidas y a la ayuda de la aplicación. Serán las personas que accedan a la aplicación a través del navegador.
\end{description}
\subsection{Casos de uso}
\begin{longtable}[H]{@{}ll@{}}
\toprule
\begin{minipage}[b]{0.23\columnwidth}\raggedright\strut
\textbf{CU-01}\strut
\end{minipage} & \begin{minipage}[b]{0.71\columnwidth}\raggedright\strut
\textbf{Registro en la aplicación}\strut
\end{minipage}\tabularnewline
\midrule
\endhead
\begin{minipage}[t]{0.23\columnwidth}\raggedright\strut
\textbf{Requisito asociado}\strut
\end{minipage} & \begin{minipage}[t]{0.71\columnwidth}\raggedright\strut
RF-1 Registro en la aplicación\strut
\end{minipage}\tabularnewline
\begin{minipage}[t]{0.23\columnwidth}\raggedright\strut
\textbf{Descripción}\strut
\end{minipage} & \begin{minipage}[t]{0.71\columnwidth}\raggedright\strut
Los usuarios que quieran solicitar el registro en la página deben rellenar un
formulario para que el súper administrador decida si acepta o declina esta
solicitud.\strut
\end{minipage}\tabularnewline
\begin{minipage}[t]{0.23\columnwidth}\raggedright\strut
\textbf{Precondición}\strut
\end{minipage} & \begin{minipage}[t]{0.71\columnwidth}\raggedright\strut
Ninguna.\strut
\end{minipage}\tabularnewline
\begin{minipage}[t]{0.23\columnwidth}\raggedright\strut
\textbf{Acciones}\strut
\end{minipage} & \begin{minipage}[t]{0.71\columnwidth}\raggedright\strut
\begin{enumerate}
\def\labelenumi{\arabic{enumi}.}
\tightlist
\item
  El usuario entra en la aplicación.
\item
  Pinchar el botón de registro de la pantalla inicial.
\item
Rellenar los campos del formulario.
\item
El sistema envía una petición al súper administrador.
\item 
El súper administrador decide si acepta o no al nuevo usuario.
\end{enumerate}\strut
\end{minipage}\tabularnewline
\begin{minipage}[t]{0.23\columnwidth}\raggedright\strut
\textbf{Postcondición}\strut
\end{minipage} & \begin{minipage}[t]{0.71\columnwidth}\raggedright\strut
Se acepta la petición de registro o se deniega.\strut
\end{minipage}\tabularnewline
\begin{minipage}[t]{0.23\columnwidth}\raggedright\strut
\textbf{Excepciones}\strut
\end{minipage} & \begin{minipage}[t]{0.71\columnwidth}\raggedright\strut
Si el administrador acepta la petición el rol del usuario cambia a
administrador/editor. \strut
\end{minipage}\tabularnewline
\begin{minipage}[t]{0.23\columnwidth}\raggedright\strut
\textbf{Importancia}\strut
\end{minipage} & \begin{minipage}[t]{0.71\columnwidth}\raggedright\strut
Alta\strut
\end{minipage}\tabularnewline
\bottomrule
\caption{CU-01 Registro en la aplicación}
\end{longtable}


\newpage
\begin{longtable}[H]{@{}ll@{}}
\toprule
\begin{minipage}[b]{0.23\columnwidth}\raggedright\strut
\textbf{CU-02}\strut
\end{minipage} & \begin{minipage}[b]{0.71\columnwidth}\raggedright\strut
\textbf{Login}\strut
\end{minipage}\tabularnewline
\midrule
\endhead
\begin{minipage}[t]{0.23\columnwidth}\raggedright\strut
\textbf{Requisito asociado}\strut
\end{minipage} & \begin{minipage}[t]{0.71\columnwidth}\raggedright\strut
RF-2 Login\strut
\end{minipage}\tabularnewline
\begin{minipage}[t]{0.23\columnwidth}\raggedright\strut
\textbf{Descripción}\strut
\end{minipage} & \begin{minipage}[t]{0.71\columnwidth}\raggedright\strut
Los administradores/editores o los súper usuarios, podrán acceder a la aplicación con sus datos de usuario.
\strut
\end{minipage}\tabularnewline
\begin{minipage}[t]{0.23\columnwidth}\raggedright\strut
\textbf{Precondición}\strut
\end{minipage} & \begin{minipage}[t]{0.71\columnwidth}\raggedright\strut
Estar registrado en la aplicación.\strut
\end{minipage}\tabularnewline
\begin{minipage}[t]{0.23\columnwidth}\raggedright\strut
\textbf{Acciones}\strut
\end{minipage} & \begin{minipage}[t]{0.71\columnwidth}\raggedright\strut
\begin{enumerate}
\def\labelenumi{\arabic{enumi}.}
\tightlist
\item
  El usuario entra en la aplicación.
\item
  Pinchar el botón de iniciar sesión.
\item
Rellenar los campos email y contraseña con sus datos de acceso.
\item
Hacer click en el botón enviar.
\end{enumerate}\strut
\end{minipage}\tabularnewline
\begin{minipage}[t]{0.23\columnwidth}\raggedright\strut
\textbf{Postcondición}\strut
\end{minipage} & \begin{minipage}[t]{0.71\columnwidth}\raggedright\strut
Se accede a la aplicación.\strut
\end{minipage}\tabularnewline
\begin{minipage}[t]{0.23\columnwidth}\raggedright\strut
\textbf{Excepciones}\strut
\end{minipage} & \begin{minipage}[t]{0.71\columnwidth}\raggedright\strut
Si se introducen datos incorrectos, se lanza un mensaje de error.
 \strut
\end{minipage}\tabularnewline
\begin{minipage}[t]{0.23\columnwidth}\raggedright\strut
\textbf{Importancia}\strut
\end{minipage} & \begin{minipage}[t]{0.71\columnwidth}\raggedright\strut
Alta\strut
\end{minipage}\tabularnewline
\bottomrule
\caption{CU-02 Login}
\end{longtable}



\newpage
\begin{longtable}[H]{@{}ll@{}}
\toprule
\begin{minipage}[b]{0.23\columnwidth}\raggedright\strut
\textbf{CU-03}\strut
\end{minipage} & \begin{minipage}[b]{0.71\columnwidth}\raggedright\strut
\textbf{Gestión de tablas: Inserción de variables}\strut
\end{minipage}\tabularnewline
\midrule
\endhead
\begin{minipage}[t]{0.23\columnwidth}\raggedright\strut
\textbf{Requisito asociado}\strut
\end{minipage} & \begin{minipage}[t]{0.71\columnwidth}\raggedright\strut
RF-3.1 Inserción de tablas\strut
\end{minipage}\tabularnewline
\begin{minipage}[t]{0.23\columnwidth}\raggedright\strut
\textbf{Descripción}\strut
\end{minipage} & \begin{minipage}[t]{0.71\columnwidth}\raggedright\strut
Los administradores/editores o los súper usuarios, podrán insertar tablas con
valores nuevos a la base de datos.
\strut
\end{minipage}\tabularnewline
\begin{minipage}[t]{0.23\columnwidth}\raggedright\strut
\textbf{Precondición}\strut
\end{minipage} & \begin{minipage}[t]{0.71\columnwidth}\raggedright\strut
Ninguna.\strut
\end{minipage}\tabularnewline
\begin{minipage}[t]{0.23\columnwidth}\raggedright\strut
\textbf{Acciones}\strut
\end{minipage} & \begin{minipage}[t]{0.71\columnwidth}\raggedright\strut
\begin{enumerate}
\def\labelenumi{\arabic{enumi}.}
\tightlist
\item
  El usuario entra en la aplicación.
\item
  Pinchar el botón de crear tabla.
\item
Rellenar los campos del formulario correspondientes al nombre de la tabla, sus categorías y sus ámbitos.
\item
Hacer click en el botón enviar.
\item 
Rellenar las celdas con los valores de la tabla.
\item
Hacer click en guardar.
\end{enumerate}\strut
\end{minipage}\tabularnewline
\begin{minipage}[t]{0.23\columnwidth}\raggedright\strut
\textbf{Postcondición}\strut
\end{minipage} & \begin{minipage}[t]{0.71\columnwidth}\raggedright\strut
Se acepta la petición de registro o se deniega.\strut
\end{minipage}\tabularnewline
\begin{minipage}[t]{0.23\columnwidth}\raggedright\strut
\textbf{Excepciones}\strut
\end{minipage} & \begin{minipage}[t]{0.71\columnwidth}\raggedright\strut
Ninguna. \strut
\end{minipage}\tabularnewline
\begin{minipage}[t]{0.23\columnwidth}\raggedright\strut
\textbf{Importancia}\strut
\end{minipage} & \begin{minipage}[t]{0.71\columnwidth}\raggedright\strut
Alta\strut
\end{minipage}\tabularnewline
\bottomrule
\caption{CU-03 Gestión de tablas: Inserción de variables}
\end{longtable}

\newpage
\begin{longtable}[H]{@{}ll@{}}
\toprule
\begin{minipage}[b]{0.23\columnwidth}\raggedright\strut
\textbf{CU-03}\strut
\end{minipage} & \begin{minipage}[b]{0.71\columnwidth}\raggedright\strut
\textbf{Gestión de tablas: Modificación de las tablas. Añadir año}\strut
\end{minipage}\tabularnewline
\midrule
\endhead
\begin{minipage}[t]{0.23\columnwidth}\raggedright\strut
\textbf{Requisitos asociados}\strut
\end{minipage} & \begin{minipage}[t]{0.71\columnwidth}\raggedright\strut
RF3, RF3.3 y RF3.3.1 Añadir año\strut
\end{minipage}\tabularnewline
\begin{minipage}[t]{0.23\columnwidth}\raggedright\strut
\textbf{Descripción}\strut
\end{minipage} & \begin{minipage}[t]{0.71\columnwidth}\raggedright\strut
En cada tabla predefinida se podrá añadir un año para continuar con el registro
de los datos.
\strut
\end{minipage}\tabularnewline
\begin{minipage}[t]{0.23\columnwidth}\raggedright\strut
\textbf{Precondición}\strut
\end{minipage} & \begin{minipage}[t]{0.71\columnwidth}\raggedright\strut
Que haya una tabla predefinida.\strut
\end{minipage}\tabularnewline
\begin{minipage}[t]{0.23\columnwidth}\raggedright\strut
\textbf{Acciones}\strut
\end{minipage} & \begin{minipage}[t]{0.71\columnwidth}\raggedright\strut
\begin{enumerate}
\def\labelenumi{\arabic{enumi}.}
\tightlist
\item
Hacer click en el botón modificar valores.
\item
Hacer click en el menú de opciones.
\item
En el apartado de añadir, pinchar en año.
\item
El sistema añadirá a la derecha de la tabla un nuevo año con
campos en el lugar de las celdas.
\item 
Rellenar el campo del año.
\item
Rellenar los valores que corresponden al nuevo año.
\item
Hacer click en enviar.
\end{enumerate}\strut
\end{minipage}\tabularnewline
\begin{minipage}[t]{0.23\columnwidth}\raggedright\strut
\textbf{Postcondición}\strut
\end{minipage} & \begin{minipage}[t]{0.71\columnwidth}\raggedright\strut
Se almacenará en el sistema los valores del nuevo año.\strut
\end{minipage}\tabularnewline
\begin{minipage}[t]{0.23\columnwidth}\raggedright\strut
\textbf{Excepciones}\strut
\end{minipage} & \begin{minipage}[t]{0.71\columnwidth}\raggedright\strut
Mensaje: Es necesario rellenar este campo si queda alguno vacío. \strut
\end{minipage}\tabularnewline
\begin{minipage}[t]{0.23\columnwidth}\raggedright\strut
\textbf{Importancia}\strut
\end{minipage} & \begin{minipage}[t]{0.71\columnwidth}\raggedright\strut
Alta\strut
\end{minipage}\tabularnewline
\bottomrule
\caption{CU-04 Gestión de tablas: Modificación de tablas. Añadir año.}
\end{longtable}

\newpage
\begin{longtable}[H]{@{}ll@{}}
\toprule
\begin{minipage}[b]{0.23\columnwidth}\raggedright\strut
\textbf{CU-05}\strut
\end{minipage} & \begin{minipage}[b]{0.71\columnwidth}\raggedright\strut
\textbf{Gestión de tablas: Modificación de las tablas. Añadir categoría}\strut
\end{minipage}\tabularnewline
\midrule
\endhead
\begin{minipage}[t]{0.23\columnwidth}\raggedright\strut
\textbf{Requisitos asociados}\strut
\end{minipage} & \begin{minipage}[t]{0.71\columnwidth}\raggedright\strut
RF3, RF3.3 y RF3.3.2 Añadir categoría\strut
\end{minipage}\tabularnewline
\begin{minipage}[t]{0.23\columnwidth}\raggedright\strut
\textbf{Descripción}\strut
\end{minipage} & \begin{minipage}[t]{0.71\columnwidth}\raggedright\strut
Se podrá añadir categorías nuevas y se podrá asignar la súper categoría a la
que pertenecen.
\strut
\end{minipage}\tabularnewline
\begin{minipage}[t]{0.23\columnwidth}\raggedright\strut
\textbf{Precondición}\strut
\end{minipage} & \begin{minipage}[t]{0.71\columnwidth}\raggedright\strut
Que haya una tabla predefinida.\strut
\end{minipage}\tabularnewline
\begin{minipage}[t]{0.23\columnwidth}\raggedright\strut
\textbf{Acciones}\strut
\end{minipage} & \begin{minipage}[t]{0.71\columnwidth}\raggedright\strut
\begin{enumerate}
\def\labelenumi{\arabic{enumi}.}
\tightlist
\item
Hacer click en el botón modificar valores.
\item
Hacer click en el menú de opciones.
\item
En el apartado de añadir, pinchar en categoría.
\item
El sistema nos redirigirá a una página con una tabla y un
formulario.
\item 
Rellenar el formulario con el nombre de la nueva categoría y
seleccionar si se desea una súper categoría para asignarle.
\item
Rellenar las celdas con los datos de la nueva categoría.
\item
Hacer click en enviar.
\end{enumerate}\strut
\end{minipage}\tabularnewline
\begin{minipage}[t]{0.23\columnwidth}\raggedright\strut
\textbf{Postcondición}\strut
\end{minipage} & \begin{minipage}[t]{0.71\columnwidth}\raggedright\strut
Se almacenará en el sistema los valores de la nueva categoría.\strut
\end{minipage}\tabularnewline
\begin{minipage}[t]{0.23\columnwidth}\raggedright\strut
\textbf{Excepciones}\strut
\end{minipage} & \begin{minipage}[t]{0.71\columnwidth}\raggedright\strut
Mensaje: Es necesario rellenar este campo si queda alguno vacío. \strut
\end{minipage}\tabularnewline
\begin{minipage}[t]{0.23\columnwidth}\raggedright\strut
\textbf{Importancia}\strut
\end{minipage} & \begin{minipage}[t]{0.71\columnwidth}\raggedright\strut
Alta\strut
\end{minipage}\tabularnewline
\bottomrule
\caption{CU-05 Gestión de tablas: Modificación de tablas. Añadir categoría.}
\end{longtable}

\newpage
\begin{longtable}[H]{@{}ll@{}}
\toprule
\begin{minipage}[b]{0.23\columnwidth}\raggedright\strut
\textbf{CU-06}\strut
\end{minipage} & \begin{minipage}[b]{0.71\columnwidth}\raggedright\strut
\textbf{Gestión de tablas: Modificación de las tablas. Añadir ámbito}\strut
\end{minipage}\tabularnewline
\midrule
\endhead
\begin{minipage}[t]{0.23\columnwidth}\raggedright\strut
\textbf{Requisitos asociados}\strut
\end{minipage} & \begin{minipage}[t]{0.71\columnwidth}\raggedright\strut
RF3, RF3.3 y RF3.3.3 Añadir ámbito\strut
\end{minipage}\tabularnewline
\begin{minipage}[t]{0.23\columnwidth}\raggedright\strut
\textbf{Descripción}\strut
\end{minipage} & \begin{minipage}[t]{0.71\columnwidth}\raggedright\strut
Las tablas predefinidas tendrán la posibilidad de añadir nuevos ámbitos
geográficos. Cada nuevo ámbito vendrá acompañado de las distintas
categorías almacenadas en la tabla.
\strut
\end{minipage}\tabularnewline
\begin{minipage}[t]{0.23\columnwidth}\raggedright\strut
\textbf{Precondición}\strut
\end{minipage} & \begin{minipage}[t]{0.71\columnwidth}\raggedright\strut
Que haya una tabla predefinida.\strut
\end{minipage}\tabularnewline
\begin{minipage}[t]{0.23\columnwidth}\raggedright\strut
\textbf{Acciones}\strut
\end{minipage} & \begin{minipage}[t]{0.71\columnwidth}\raggedright\strut
\begin{enumerate}
\def\labelenumi{\arabic{enumi}.}
\tightlist
\item
Hacer click en el botón modificar valores.
\item
Hacer click en el menú de opciones.
\item
En el apartado de añadir, pinchar en ámbito.
\item
El sistema añadirá en la parte inferior celdas por valor de todas las
categorías que pertenecen a la tabla.
\item 
Rellenar el campo del nombre del ámbito
\item
Rellenar las celdas con los datos del nuevo ámbito
\item
Hacer click en enviar.
\end{enumerate}\strut
\end{minipage}\tabularnewline
\begin{minipage}[t]{0.23\columnwidth}\raggedright\strut
\textbf{Postcondición}\strut
\end{minipage} & \begin{minipage}[t]{0.71\columnwidth}\raggedright\strut
Se almacenará en el sistema los valores del nuevo ámbito.\strut
\end{minipage}\tabularnewline
\begin{minipage}[t]{0.23\columnwidth}\raggedright\strut
\textbf{Excepciones}\strut
\end{minipage} & \begin{minipage}[t]{0.71\columnwidth}\raggedright\strut
Mensaje: Es necesario rellenar este campo si queda alguno vacío. \strut
\end{minipage}\tabularnewline
\begin{minipage}[t]{0.23\columnwidth}\raggedright\strut
\textbf{Importancia}\strut
\end{minipage} & \begin{minipage}[t]{0.71\columnwidth}\raggedright\strut
Alta\strut
\end{minipage}\tabularnewline
\bottomrule
\caption{CU-06 Gestión de tablas: Modificación de tablas. Añadir ámbito.}
\end{longtable}

\newpage
\begin{longtable}[H]{@{}ll@{}}
\toprule
\begin{minipage}[b]{0.23\columnwidth}\raggedright\strut
\textbf{CU-07}\strut
\end{minipage} & \begin{minipage}[b]{0.71\columnwidth}\raggedright\strut
\textbf{Gestión de tablas: Modificación de las tablas. Borrar año}\strut
\end{minipage}\tabularnewline
\midrule
\endhead
\begin{minipage}[t]{0.23\columnwidth}\raggedright\strut
\textbf{Requisitos asociados}\strut
\end{minipage} & \begin{minipage}[t]{0.71\columnwidth}\raggedright\strut
RF3, RF3.3 y RF3.3.4 Borrar año\strut
\end{minipage}\tabularnewline
\begin{minipage}[t]{0.23\columnwidth}\raggedright\strut
\textbf{Descripción}\strut
\end{minipage} & \begin{minipage}[t]{0.71\columnwidth}\raggedright\strut
Se podrá borrar años de una tabla.
\strut
\end{minipage}\tabularnewline
\begin{minipage}[t]{0.23\columnwidth}\raggedright\strut
\textbf{Precondición}\strut
\end{minipage} & \begin{minipage}[t]{0.71\columnwidth}\raggedright\strut
Que haya una tabla predefinida.\strut
\end{minipage}\tabularnewline
\begin{minipage}[t]{0.23\columnwidth}\raggedright\strut
\textbf{Acciones}\strut
\end{minipage} & \begin{minipage}[t]{0.71\columnwidth}\raggedright\strut
\begin{enumerate}
\def\labelenumi{\arabic{enumi}.}
\tightlist
\item
Hacer click en el botón modificar valores.
\item
Hacer click en el menú de opciones.
\item
En el apartado de borrar, pinchar en año.
\item
El sistema redirigirá a una página donde existirá un formulario
\item 
Seleccionar los años que se desean borrar.
\item
Hacer click en enviar.
\end{enumerate}\strut
\end{minipage}\tabularnewline
\begin{minipage}[t]{0.23\columnwidth}\raggedright\strut
\textbf{Postcondición}\strut
\end{minipage} & \begin{minipage}[t]{0.71\columnwidth}\raggedright\strut
Se borrarán los datos del año en la base de datos.\strut
\end{minipage}\tabularnewline
\begin{minipage}[t]{0.23\columnwidth}\raggedright\strut
\textbf{Excepciones}\strut
\end{minipage} & \begin{minipage}[t]{0.71\columnwidth}\raggedright\strut
Ninguna. \strut
\end{minipage}\tabularnewline
\begin{minipage}[t]{0.23\columnwidth}\raggedright\strut
\textbf{Importancia}\strut
\end{minipage} & \begin{minipage}[t]{0.71\columnwidth}\raggedright\strut
Alta\strut
\end{minipage}\tabularnewline
\bottomrule
\caption{CU-07 Gestión de tablas: Modificación de tablas. Borrar año.}
\end{longtable}

\newpage
\begin{longtable}[H]{@{}ll@{}}
\toprule
\begin{minipage}[b]{0.23\columnwidth}\raggedright\strut
\textbf{CU-08}\strut
\end{minipage} & \begin{minipage}[b]{0.71\columnwidth}\raggedright\strut
\textbf{Gestión de tablas: Modificación de las tablas. Borrar categoría}\strut
\end{minipage}\tabularnewline
\midrule
\endhead
\begin{minipage}[t]{0.23\columnwidth}\raggedright\strut
\textbf{Requisitos asociados}\strut
\end{minipage} & \begin{minipage}[t]{0.71\columnwidth}\raggedright\strut
RF3, RF3.3 y RF3.3.5 Borrar categoría\strut
\end{minipage}\tabularnewline
\begin{minipage}[t]{0.23\columnwidth}\raggedright\strut
\textbf{Descripción}\strut
\end{minipage} & \begin{minipage}[t]{0.71\columnwidth}\raggedright\strut
Se podrá borrar las categorías de una tabla.
\strut
\end{minipage}\tabularnewline
\begin{minipage}[t]{0.23\columnwidth}\raggedright\strut
\textbf{Precondición}\strut
\end{minipage} & \begin{minipage}[t]{0.71\columnwidth}\raggedright\strut
Que haya una tabla predefinida.\strut
\end{minipage}\tabularnewline
\begin{minipage}[t]{0.23\columnwidth}\raggedright\strut
\textbf{Acciones}\strut
\end{minipage} & \begin{minipage}[t]{0.71\columnwidth}\raggedright\strut
\begin{enumerate}
\def\labelenumi{\arabic{enumi}.}
\tightlist
\item
Hacer click en el botón modificar valores.
\item
Hacer click en el menú de opciones.
\item
En el apartado de borrar, pinchar en categoría.
\item
El sistema redirigirá a una página donde existirá un formulario
\item 
Seleccionar las categorías que se desean borrar.
\item
Hacer click en enviar.
\end{enumerate}\strut
\end{minipage}\tabularnewline
\begin{minipage}[t]{0.23\columnwidth}\raggedright\strut
\textbf{Postcondición}\strut
\end{minipage} & \begin{minipage}[t]{0.71\columnwidth}\raggedright\strut
Se borrarán los datos de la categoría en la base de datos.\strut
\end{minipage}\tabularnewline
\begin{minipage}[t]{0.23\columnwidth}\raggedright\strut
\textbf{Excepciones}\strut
\end{minipage} & \begin{minipage}[t]{0.71\columnwidth}\raggedright\strut
Ninguna. \strut
\end{minipage}\tabularnewline
\begin{minipage}[t]{0.23\columnwidth}\raggedright\strut
\textbf{Importancia}\strut
\end{minipage} & \begin{minipage}[t]{0.71\columnwidth}\raggedright\strut
Alta\strut
\end{minipage}\tabularnewline
\bottomrule
\caption{CU-08 Gestión de tablas: Modificación de tablas. Borrar categoría.}
\end{longtable}

\newpage
\begin{longtable}[H]{@{}ll@{}}
\toprule
\begin{minipage}[b]{0.23\columnwidth}\raggedright\strut
\textbf{CU-09}\strut
\end{minipage} & \begin{minipage}[b]{0.71\columnwidth}\raggedright\strut
\textbf{Gestión de tablas: Modificación de las tablas. Borrar ámbito}\strut
\end{minipage}\tabularnewline
\midrule
\endhead
\begin{minipage}[t]{0.23\columnwidth}\raggedright\strut
\textbf{Requisitos asociados}\strut
\end{minipage} & \begin{minipage}[t]{0.71\columnwidth}\raggedright\strut
RF3, RF3.3 y RF3.3.6 Borrar ámbito\strut
\end{minipage}\tabularnewline
\begin{minipage}[t]{0.23\columnwidth}\raggedright\strut
\textbf{Descripción}\strut
\end{minipage} & \begin{minipage}[t]{0.71\columnwidth}\raggedright\strut
Se podrá borrar un ámbito geográfico de una tabla.
\strut
\end{minipage}\tabularnewline
\begin{minipage}[t]{0.23\columnwidth}\raggedright\strut
\textbf{Precondición}\strut
\end{minipage} & \begin{minipage}[t]{0.71\columnwidth}\raggedright\strut
Que haya una tabla predefinida.\strut
\end{minipage}\tabularnewline
\begin{minipage}[t]{0.23\columnwidth}\raggedright\strut
\textbf{Acciones}\strut
\end{minipage} & \begin{minipage}[t]{0.71\columnwidth}\raggedright\strut
\begin{enumerate}
\def\labelenumi{\arabic{enumi}.}
\tightlist
\item
Hacer click en el botón modificar valores.
\item
Hacer click en el menú de opciones.
\item
En el apartado de borrar, pinchar en ámbito.
\item
El sistema redirigirá a una página donde existirá un formulario
\item 
Seleccionar los ámbitos que se desean borrar.
\item
Hacer click en enviar.
\end{enumerate}\strut
\end{minipage}\tabularnewline
\begin{minipage}[t]{0.23\columnwidth}\raggedright\strut
\textbf{Postcondición}\strut
\end{minipage} & \begin{minipage}[t]{0.71\columnwidth}\raggedright\strut
Se borrarán los datos del ámbito en la base de datos.\strut
\end{minipage}\tabularnewline
\begin{minipage}[t]{0.23\columnwidth}\raggedright\strut
\textbf{Excepciones}\strut
\end{minipage} & \begin{minipage}[t]{0.71\columnwidth}\raggedright\strut
Ninguna. \strut
\end{minipage}\tabularnewline
\begin{minipage}[t]{0.23\columnwidth}\raggedright\strut
\textbf{Importancia}\strut
\end{minipage} & \begin{minipage}[t]{0.71\columnwidth}\raggedright\strut
Alta\strut
\end{minipage}\tabularnewline
\bottomrule
\caption{CU-09 Gestión de tablas: Modificación de tablas. Borrar ámbito.}
\end{longtable}

\newpage
\begin{longtable}[H]{@{}ll@{}}
\toprule
\begin{minipage}[b]{0.23\columnwidth}\raggedright\strut
\textbf{CU-10}\strut
\end{minipage} & \begin{minipage}[b]{0.71\columnwidth}\raggedright\strut
\textbf{Gestión de tablas: Visualización}\strut
\end{minipage}\tabularnewline
\midrule
\endhead
\begin{minipage}[t]{0.23\columnwidth}\raggedright\strut
\textbf{Requisitos asociados}\strut
\end{minipage} & \begin{minipage}[t]{0.71\columnwidth}\raggedright\strut
RF3.4 Visualización de tablas\strut
\end{minipage}\tabularnewline
\begin{minipage}[t]{0.23\columnwidth}\raggedright\strut
\textbf{Descripción}\strut
\end{minipage} & \begin{minipage}[t]{0.71\columnwidth}\raggedright\strut
Los datos almacenados en la base de datos se mostrarán al usuario en forma
de tablas.
\strut
\end{minipage}\tabularnewline
\begin{minipage}[t]{0.23\columnwidth}\raggedright\strut
\textbf{Precondición}\strut
\end{minipage} & \begin{minipage}[t]{0.71\columnwidth}\raggedright\strut
Que haya una tabla predefinida.\strut
\end{minipage}\tabularnewline
\begin{minipage}[t]{0.23\columnwidth}\raggedright\strut
\textbf{Acciones}\strut
\end{minipage} & \begin{minipage}[t]{0.71\columnwidth}\raggedright\strut
\begin{enumerate}
\def\labelenumi{\arabic{enumi}.}
\tightlist
\item
Pinchar en el menú situado a la izquierda de la pantalla, en el
enlace a tablas.
\item
Seleccionar el enlace a Tablas predefinidas.
\item
Seleccionar la tabla que se desea ver.
\item
El sistema redirigirá a una página donde existirá un formulario para seleccionar los datos se quieran ver.
\item 
Rellenar los campos del formulario.
\item
Hacer click en enviar.
\end{enumerate}\strut
\end{minipage}\tabularnewline
\begin{minipage}[t]{0.23\columnwidth}\raggedright\strut
\textbf{Postcondición}\strut
\end{minipage} & \begin{minipage}[t]{0.71\columnwidth}\raggedright\strut
Se mostrará la tabla con los valores elegidos.\strut
\end{minipage}\tabularnewline
\begin{minipage}[t]{0.23\columnwidth}\raggedright\strut
\textbf{Excepciones}\strut
\end{minipage} & \begin{minipage}[t]{0.71\columnwidth}\raggedright\strut
Ninguna. \strut
\end{minipage}\tabularnewline
\begin{minipage}[t]{0.23\columnwidth}\raggedright\strut
\textbf{Importancia}\strut
\end{minipage} & \begin{minipage}[t]{0.71\columnwidth}\raggedright\strut
Alta\strut
\end{minipage}\tabularnewline
\bottomrule
\caption{CU-10 Gestión de tablas: Visualización.}
\end{longtable}

\newpage
\begin{longtable}[H]{@{}ll@{}}
\toprule
\begin{minipage}[b]{0.23\columnwidth}\raggedright\strut
\textbf{CU-11}\strut
\end{minipage} & \begin{minipage}[b]{0.71\columnwidth}\raggedright\strut
\textbf{Gestión de tablas: Borrado de tablas}\strut
\end{minipage}\tabularnewline
\midrule
\endhead
\begin{minipage}[t]{0.23\columnwidth}\raggedright\strut
\textbf{Requisitos asociados}\strut
\end{minipage} & \begin{minipage}[t]{0.71\columnwidth}\raggedright\strut
RF3 y R3.5 Borrado de tablas\strut
\end{minipage}\tabularnewline
\begin{minipage}[t]{0.23\columnwidth}\raggedright\strut
\textbf{Descripción}\strut
\end{minipage} & \begin{minipage}[t]{0.71\columnwidth}\raggedright\strut
Borrado de las tablas de la aplicación.
\strut
\end{minipage}\tabularnewline
\begin{minipage}[t]{0.23\columnwidth}\raggedright\strut
\textbf{Precondición}\strut
\end{minipage} & \begin{minipage}[t]{0.71\columnwidth}\raggedright\strut
Que haya una tabla predefinida.\strut
\end{minipage}\tabularnewline
\begin{minipage}[t]{0.23\columnwidth}\raggedright\strut
\textbf{Acciones}\strut
\end{minipage} & \begin{minipage}[t]{0.71\columnwidth}\raggedright\strut
\begin{enumerate}
\def\labelenumi{\arabic{enumi}.}
\tightlist
\item
Pinchar en el menú situado a la izquierda de la pantalla, en el
enlace a tablas.
\item
Seleccionar el enlace a Tablas predefinidas.
\item
Seleccionar la tabla que se desea borrar.
\item
Hacer click en aceptar.
\end{enumerate}\strut
\end{minipage}\tabularnewline
\begin{minipage}[t]{0.23\columnwidth}\raggedright\strut
\textbf{Postcondición}\strut
\end{minipage} & \begin{minipage}[t]{0.71\columnwidth}\raggedright\strut
Se borrará la tabla de la base de datos.\strut
\end{minipage}\tabularnewline
\begin{minipage}[t]{0.23\columnwidth}\raggedright\strut
\textbf{Excepciones}\strut
\end{minipage} & \begin{minipage}[t]{0.71\columnwidth}\raggedright\strut
Ninguna. \strut
\end{minipage}\tabularnewline
\begin{minipage}[t]{0.23\columnwidth}\raggedright\strut
\textbf{Importancia}\strut
\end{minipage} & \begin{minipage}[t]{0.71\columnwidth}\raggedright\strut
Alta\strut
\end{minipage}\tabularnewline
\bottomrule
\caption{CU-11 Gestión de tablas: Borrado de tablas.}
\end{longtable}

\newpage
\begin{longtable}[H]{@{}ll@{}}
\toprule
\begin{minipage}[b]{0.23\columnwidth}\raggedright\strut
\textbf{CU-12}\strut
\end{minipage} & \begin{minipage}[b]{0.71\columnwidth}\raggedright\strut
\textbf{Creado de gráficos}\strut
\end{minipage}\tabularnewline
\midrule
\endhead
\begin{minipage}[t]{0.23\columnwidth}\raggedright\strut
\textbf{Requisitos asociados}\strut
\end{minipage} & \begin{minipage}[t]{0.71\columnwidth}\raggedright\strut
RF4 Creación de gráficos\strut
\end{minipage}\tabularnewline
\begin{minipage}[t]{0.23\columnwidth}\raggedright\strut
\textbf{Descripción}\strut
\end{minipage} & \begin{minipage}[t]{0.71\columnwidth}\raggedright\strut
Se generarán gráficos automáticamente desde los datos introducidos en la
tabla, se podrá filtrar los datos que se desean ver en el gráfico por medio de un
formulario.
\strut
\end{minipage}\tabularnewline
\begin{minipage}[t]{0.23\columnwidth}\raggedright\strut
\textbf{Precondición}\strut
\end{minipage} & \begin{minipage}[t]{0.71\columnwidth}\raggedright\strut
Que haya una tabla predefinida.\strut
\end{minipage}\tabularnewline
\begin{minipage}[t]{0.23\columnwidth}\raggedright\strut
\textbf{Acciones}\strut
\end{minipage} & \begin{minipage}[t]{0.71\columnwidth}\raggedright\strut
\begin{enumerate}
\def\labelenumi{\arabic{enumi}.}
\tightlist
\item
Pinchar en el menú situado a la izquierda de la pantalla, en el
enlace a tablas.
\item
Seleccionar el enlace a Tablas predefinidas.
\item
Seleccionar la tabla que se desea ver.
\item
El sistema enviará a una página con un formulario para filtrar los
datos. 
\item
Rellenar los campos del formulario.
\item
Hacer click en enviar.
\item 
El sistema mostrará un gráfico de barras con todos los valores de
la tabla.
\item
El sistema mostrará un formulario para filtrar los datos que se
desea mostrar en el formulario.
\item
Rellenar los campos del formulario.
\item
Hacer click en enviar.
\end{enumerate}\strut
\end{minipage}\tabularnewline
\begin{minipage}[t]{0.23\columnwidth}\raggedright\strut
\textbf{Postcondición}\strut
\end{minipage} & \begin{minipage}[t]{0.71\columnwidth}\raggedright\strut
Se mostrarán los datos en gráficos de barras,lineas o radar.\strut
\end{minipage}\tabularnewline
\begin{minipage}[t]{0.23\columnwidth}\raggedright\strut
\textbf{Excepciones}\strut
\end{minipage} & \begin{minipage}[t]{0.71\columnwidth}\raggedright\strut
Se deberán rellenar todos los datos. \strut
\end{minipage}\tabularnewline
\begin{minipage}[t]{0.23\columnwidth}\raggedright\strut
\textbf{Importancia}\strut
\end{minipage} & \begin{minipage}[t]{0.71\columnwidth}\raggedright\strut
Alta\strut
\end{minipage}\tabularnewline
\bottomrule
\caption{CU-12 Creado de gráficos}
\end{longtable}

\newpage
\begin{longtable}[H]{@{}ll@{}}
\toprule
\begin{minipage}[b]{0.23\columnwidth}\raggedright\strut
\textbf{CU-13}\strut
\end{minipage} & \begin{minipage}[b]{0.71\columnwidth}\raggedright\strut
\textbf{Gestión de datos: Variables. Modificar}\strut
\end{minipage}\tabularnewline
\midrule
\endhead
\begin{minipage}[t]{0.23\columnwidth}\raggedright\strut
\textbf{Requisitos asociados}\strut
\end{minipage} & \begin{minipage}[t]{0.71\columnwidth}\raggedright\strut
RF5.1.1 Modificar variable\strut
\end{minipage}\tabularnewline
\begin{minipage}[t]{0.23\columnwidth}\raggedright\strut
\textbf{Descripción}\strut
\end{minipage} & \begin{minipage}[t]{0.71\columnwidth}\raggedright\strut
Se podrá modificar el nombre, la fuente, el tipo y la descripción
\strut
\end{minipage}\tabularnewline
\begin{minipage}[t]{0.23\columnwidth}\raggedright\strut
\textbf{Precondición}\strut
\end{minipage} & \begin{minipage}[t]{0.71\columnwidth}\raggedright\strut
Que haya una tabla predefinida y estar registrado en la aplicación como Administrador/Editor o Súper administrador.\strut
\end{minipage}\tabularnewline
\begin{minipage}[t]{0.23\columnwidth}\raggedright\strut
\textbf{Acciones}\strut
\end{minipage} & \begin{minipage}[t]{0.71\columnwidth}\raggedright\strut
\begin{enumerate}
\def\labelenumi{\arabic{enumi}.}
\tightlist
\item
Pinchar en el menú situado a la izquierda de la pantalla, en el
enlace a Gestión de datos.
\item
Seleccionar Variables.
\item
El sistema mostrará las variables almacenadas en el sistema.
\item
Seleccionar la variable que se desea modificar.
\item
El sistema enviará a una página con un formulario para modificar
la tabla.
\item
Rellenar los datos que se desean modificar
\item
Hacer click en modificar.
\end{enumerate}\strut
\end{minipage}\tabularnewline
\begin{minipage}[t]{0.23\columnwidth}\raggedright\strut
\textbf{Postcondición}\strut
\end{minipage} & \begin{minipage}[t]{0.71\columnwidth}\raggedright\strut
Se modificará la variable seleccionada.\strut
\end{minipage}\tabularnewline
\begin{minipage}[t]{0.23\columnwidth}\raggedright\strut
\textbf{Excepciones}\strut
\end{minipage} & \begin{minipage}[t]{0.71\columnwidth}\raggedright\strut
Ninguna. \strut
\end{minipage}\tabularnewline
\begin{minipage}[t]{0.23\columnwidth}\raggedright\strut
\textbf{Importancia}\strut
\end{minipage} & \begin{minipage}[t]{0.71\columnwidth}\raggedright\strut
Alta\strut
\end{minipage}\tabularnewline
\bottomrule
\caption{CU-13 Gestión de variables. Modificar}
\end{longtable}

\newpage
\begin{longtable}[H]{@{}ll@{}}
\toprule
\begin{minipage}[b]{0.23\columnwidth}\raggedright\strut
\textbf{CU-14}\strut
\end{minipage} & \begin{minipage}[b]{0.71\columnwidth}\raggedright\strut
\textbf{Gestión de datos: Variables. Borrar}\strut
\end{minipage}\tabularnewline
\midrule
\endhead
\begin{minipage}[t]{0.23\columnwidth}\raggedright\strut
\textbf{Requisitos asociados}\strut
\end{minipage} & \begin{minipage}[t]{0.71\columnwidth}\raggedright\strut
RF5.1.2 Borrar variable\strut
\end{minipage}\tabularnewline
\begin{minipage}[t]{0.23\columnwidth}\raggedright\strut
\textbf{Descripción}\strut
\end{minipage} & \begin{minipage}[t]{0.71\columnwidth}\raggedright\strut
Se podrá borrar una variable
\strut
\end{minipage}\tabularnewline
\begin{minipage}[t]{0.23\columnwidth}\raggedright\strut
\textbf{Precondición}\strut
\end{minipage} & \begin{minipage}[t]{0.71\columnwidth}\raggedright\strut
Que haya una tabla predefinida y estar registrado en la aplicación como Administrador/Editor o Súper administrador.\strut
\end{minipage}\tabularnewline
\begin{minipage}[t]{0.23\columnwidth}\raggedright\strut
\textbf{Acciones}\strut
\end{minipage} & \begin{minipage}[t]{0.71\columnwidth}\raggedright\strut
\begin{enumerate}
\def\labelenumi{\arabic{enumi}.}
\tightlist
\item
Pinchar en el menú situado a la izquierda de la pantalla, en el
enlace a Gestión de datos.
\item
Seleccionar Variables.
\item
El sistema mostrará las variables almacenadas en el sistema.
\item
Seleccionar la variable que se desea borrar.
\item
Hacer click en aceptar.
\end{enumerate}\strut
\end{minipage}\tabularnewline
\begin{minipage}[t]{0.23\columnwidth}\raggedright\strut
\textbf{Postcondición}\strut
\end{minipage} & \begin{minipage}[t]{0.71\columnwidth}\raggedright\strut
Se borrará la variable seleccionada.\strut
\end{minipage}\tabularnewline
\begin{minipage}[t]{0.23\columnwidth}\raggedright\strut
\textbf{Excepciones}\strut
\end{minipage} & \begin{minipage}[t]{0.71\columnwidth}\raggedright\strut
Ninguna. \strut
\end{minipage}\tabularnewline
\begin{minipage}[t]{0.23\columnwidth}\raggedright\strut
\textbf{Importancia}\strut
\end{minipage} & \begin{minipage}[t]{0.71\columnwidth}\raggedright\strut
Alta\strut
\end{minipage}\tabularnewline
\bottomrule
\caption{CU-14 Gestión de variables. Borrar}
\end{longtable}

\newpage
\begin{longtable}[H]{@{}ll@{}}
\toprule
\begin{minipage}[b]{0.23\columnwidth}\raggedright\strut
\textbf{CU-15}\strut
\end{minipage} & \begin{minipage}[b]{0.71\columnwidth}\raggedright\strut
\textbf{Gestión de datos: Categorías. Crear}\strut
\end{minipage}\tabularnewline
\midrule
\endhead
\begin{minipage}[t]{0.23\columnwidth}\raggedright\strut
\textbf{Requisitos asociados}\strut
\end{minipage} & \begin{minipage}[t]{0.71\columnwidth}\raggedright\strut
RF5.2.1 Crear categoría.\strut
\end{minipage}\tabularnewline
\begin{minipage}[t]{0.23\columnwidth}\raggedright\strut
\textbf{Descripción}\strut
\end{minipage} & \begin{minipage}[t]{0.71\columnwidth}\raggedright\strut
Se podrá crear una nueva categoría.
\strut
\end{minipage}\tabularnewline
\begin{minipage}[t]{0.23\columnwidth}\raggedright\strut
\textbf{Precondición}\strut
\end{minipage} & \begin{minipage}[t]{0.71\columnwidth}\raggedright\strut
Estar registrado en la aplicación como Administrador/Editor o Súper administrador.\strut
\end{minipage}\tabularnewline
\begin{minipage}[t]{0.23\columnwidth}\raggedright\strut
\textbf{Acciones}\strut
\end{minipage} & \begin{minipage}[t]{0.71\columnwidth}\raggedright\strut
\begin{enumerate}
\def\labelenumi{\arabic{enumi}.}
\tightlist
\item
Pinchar en el menú situado a la izquierda de la pantalla, en el
enlace a Gestión de datos.
\item
Seleccionar Categorías.
\item
El sistema mostrará las categorías almacenadas en el sistema.
\item
Hacer click en el botón Crear categoría
\item
El sistema enviará a una página con un formulario para crear la
nueva categoría.
\item
Rellenar el formulario.
\item
Hacer click en Crear.
\end{enumerate}\strut
\end{minipage}\tabularnewline
\begin{minipage}[t]{0.23\columnwidth}\raggedright\strut
\textbf{Postcondición}\strut
\end{minipage} & \begin{minipage}[t]{0.71\columnwidth}\raggedright\strut
Se creará la categoría.\strut
\end{minipage}\tabularnewline
\begin{minipage}[t]{0.23\columnwidth}\raggedright\strut
\textbf{Excepciones}\strut
\end{minipage} & \begin{minipage}[t]{0.71\columnwidth}\raggedright\strut
Ninguna. \strut
\end{minipage}\tabularnewline
\begin{minipage}[t]{0.23\columnwidth}\raggedright\strut
\textbf{Importancia}\strut
\end{minipage} & \begin{minipage}[t]{0.71\columnwidth}\raggedright\strut
Alta\strut
\end{minipage}\tabularnewline
\bottomrule
\caption{CU-15 Gestión de categorías. Crear}
\end{longtable}

\newpage
\begin{longtable}[H]{@{}ll@{}}
\toprule
\begin{minipage}[b]{0.23\columnwidth}\raggedright\strut
\textbf{CU-16}\strut
\end{minipage} & \begin{minipage}[b]{0.71\columnwidth}\raggedright\strut
\textbf{Gestión de datos: Categorías. Modificar}\strut
\end{minipage}\tabularnewline
\midrule
\endhead
\begin{minipage}[t]{0.23\columnwidth}\raggedright\strut
\textbf{Requisitos asociados}\strut
\end{minipage} & \begin{minipage}[t]{0.71\columnwidth}\raggedright\strut
RF5.2.2 Modificar categoría\strut
\end{minipage}\tabularnewline
\begin{minipage}[t]{0.23\columnwidth}\raggedright\strut
\textbf{Descripción}\strut
\end{minipage} & \begin{minipage}[t]{0.71\columnwidth}\raggedright\strut
Se podrá modificar la categoría.
\strut
\end{minipage}\tabularnewline
\begin{minipage}[t]{0.23\columnwidth}\raggedright\strut
\textbf{Precondición}\strut
\end{minipage} & \begin{minipage}[t]{0.71\columnwidth}\raggedright\strut
Estar registrado en la aplicación como Administrador/Editor o Súper administrador.\strut
\end{minipage}\tabularnewline
\begin{minipage}[t]{0.23\columnwidth}\raggedright\strut
\textbf{Acciones}\strut
\end{minipage} & \begin{minipage}[t]{0.71\columnwidth}\raggedright\strut
\begin{enumerate}
\def\labelenumi{\arabic{enumi}.}
\tightlist
\item
Pinchar en el menú situado a la izquierda de la pantalla, en el
enlace a Gestión de datos.
\item
Seleccionar Categorías.
\item
El sistema mostrará las categorías almacenadas en el sistema.
\item
Seleccionar la categoría que se desea modificar.
\item
El sistema enviará a una página con un formulario para modificar
la categoría.
\item
Rellenar los datos que se desean modificar
\item
Hacer click en modificar.
\end{enumerate}\strut
\end{minipage}\tabularnewline
\begin{minipage}[t]{0.23\columnwidth}\raggedright\strut
\textbf{Postcondición}\strut
\end{minipage} & \begin{minipage}[t]{0.71\columnwidth}\raggedright\strut
Se modificará la categoría seleccionada.\strut
\end{minipage}\tabularnewline
\begin{minipage}[t]{0.23\columnwidth}\raggedright\strut
\textbf{Excepciones}\strut
\end{minipage} & \begin{minipage}[t]{0.71\columnwidth}\raggedright\strut
Ninguna. \strut
\end{minipage}\tabularnewline
\begin{minipage}[t]{0.23\columnwidth}\raggedright\strut
\textbf{Importancia}\strut
\end{minipage} & \begin{minipage}[t]{0.71\columnwidth}\raggedright\strut
Alta\strut
\end{minipage}\tabularnewline
\bottomrule
\caption{CU-16 Gestión de categorías. Modificar}
\end{longtable}

\newpage
\begin{longtable}[H]{@{}ll@{}}
\toprule
\begin{minipage}[b]{0.23\columnwidth}\raggedright\strut
\textbf{CU-17}\strut
\end{minipage} & \begin{minipage}[b]{0.71\columnwidth}\raggedright\strut
\textbf{Gestión de categorías. Borrar}\strut
\end{minipage}\tabularnewline
\midrule
\endhead
\begin{minipage}[t]{0.23\columnwidth}\raggedright\strut
\textbf{Requisitos asociados}\strut
\end{minipage} & \begin{minipage}[t]{0.71\columnwidth}\raggedright\strut
RF5.2.3 Borrar categoría\strut
\end{minipage}\tabularnewline
\begin{minipage}[t]{0.23\columnwidth}\raggedright\strut
\textbf{Descripción}\strut
\end{minipage} & \begin{minipage}[t]{0.71\columnwidth}\raggedright\strut
Se podrá borrar una categoría.
\strut
\end{minipage}\tabularnewline
\begin{minipage}[t]{0.23\columnwidth}\raggedright\strut
\textbf{Precondición}\strut
\end{minipage} & \begin{minipage}[t]{0.71\columnwidth}\raggedright\strut
Estar registrado en la aplicación como Administrador/Editor o Súper administrador.\strut
\end{minipage}\tabularnewline
\begin{minipage}[t]{0.23\columnwidth}\raggedright\strut
\textbf{Acciones}\strut
\end{minipage} & \begin{minipage}[t]{0.71\columnwidth}\raggedright\strut
\begin{enumerate}
\def\labelenumi{\arabic{enumi}.}
\tightlist
\item
Pinchar en el menú situado a la izquierda de la pantalla, en el
enlace a Gestión de datos.
\item
Seleccionar Categorías.
\item
El sistema mostrará las categorías almacenadas en el sistema.
\item
Seleccionar la categoría que se desea borrar. Se tendrá que elegir entre
\begin{enumerate}
    \item Borrar categoría de una tabla.
    \item Borrar categoría del sistema.
\end{enumerate}
\item
Hacer click en borrar.
\end{enumerate}\strut
\end{minipage}\tabularnewline
\begin{minipage}[t]{0.23\columnwidth}\raggedright\strut
\textbf{Postcondición}\strut
\end{minipage} & \begin{minipage}[t]{0.71\columnwidth}\raggedright\strut
Se borrará la categoría seleccionada.\strut
\end{minipage}\tabularnewline
\begin{minipage}[t]{0.23\columnwidth}\raggedright\strut
\textbf{Excepciones}\strut
\end{minipage} & \begin{minipage}[t]{0.71\columnwidth}\raggedright\strut
Ninguna. \strut
\end{minipage}\tabularnewline
\begin{minipage}[t]{0.23\columnwidth}\raggedright\strut
\textbf{Importancia}\strut
\end{minipage} & \begin{minipage}[t]{0.71\columnwidth}\raggedright\strut
Alta\strut
\end{minipage}\tabularnewline
\bottomrule
\caption{CU-17 Gestión de categorías. Borrar}
\end{longtable}

\newpage
\begin{longtable}[H]{@{}ll@{}}
\toprule
\begin{minipage}[b]{0.23\columnwidth}\raggedright\strut
\textbf{CU-18}\strut
\end{minipage} & \begin{minipage}[b]{0.71\columnwidth}\raggedright\strut
\textbf{Gestión de datos: Súper categorías. Crear}\strut
\end{minipage}\tabularnewline
\midrule
\endhead
\begin{minipage}[t]{0.23\columnwidth}\raggedright\strut
\textbf{Requisitos asociados}\strut
\end{minipage} & \begin{minipage}[t]{0.71\columnwidth}\raggedright\strut
RF5.3.1 Crear súper categoría.\strut
\end{minipage}\tabularnewline
\begin{minipage}[t]{0.23\columnwidth}\raggedright\strut
\textbf{Descripción}\strut
\end{minipage} & \begin{minipage}[t]{0.71\columnwidth}\raggedright\strut
Se creará una súper categoría nueva y se le podrá asignar las categorías que
están sin categoría.
\strut
\end{minipage}\tabularnewline
\begin{minipage}[t]{0.23\columnwidth}\raggedright\strut
\textbf{Precondición}\strut
\end{minipage} & \begin{minipage}[t]{0.71\columnwidth}\raggedright\strut
Estar registrado en la aplicación como Administrador/Editor o Súper administrador.\strut
\end{minipage}\tabularnewline
\begin{minipage}[t]{0.23\columnwidth}\raggedright\strut
\textbf{Acciones}\strut
\end{minipage} & \begin{minipage}[t]{0.71\columnwidth}\raggedright\strut
\begin{enumerate}
\def\labelenumi{\arabic{enumi}.}
\tightlist
\item
Pinchar en el menú situado a la izquierda de la pantalla, en el
enlace a Gestión de datos.
\item
Seleccionar súper categorías.
\item
El sistema mostrará las súper categorías almacenadas en el sistema.
\item
Hacer click en el botón Crear súper categoría
\item
El sistema enviará a una página con un formulario para crear la
nueva súper categoría.
\item
Rellenar el formulario.
\item
Hacer click en Crear.
\end{enumerate}\strut
\end{minipage}\tabularnewline
\begin{minipage}[t]{0.23\columnwidth}\raggedright\strut
\textbf{Postcondición}\strut
\end{minipage} & \begin{minipage}[t]{0.71\columnwidth}\raggedright\strut
Se creará la súper categoría.\strut
\end{minipage}\tabularnewline
\begin{minipage}[t]{0.23\columnwidth}\raggedright\strut
\textbf{Excepciones}\strut
\end{minipage} & \begin{minipage}[t]{0.71\columnwidth}\raggedright\strut
Ninguna. \strut
\end{minipage}\tabularnewline
\begin{minipage}[t]{0.23\columnwidth}\raggedright\strut
\textbf{Importancia}\strut
\end{minipage} & \begin{minipage}[t]{0.71\columnwidth}\raggedright\strut
Alta\strut
\end{minipage}\tabularnewline
\bottomrule
\caption{CU-18 Gestión de súper categorías. Crear}
\end{longtable}

\newpage
\begin{longtable}[H]{@{}ll@{}}
\toprule
\begin{minipage}[b]{0.23\columnwidth}\raggedright\strut
\textbf{CU-19}\strut
\end{minipage} & \begin{minipage}[b]{0.71\columnwidth}\raggedright\strut
\textbf{Gestión de datos:Súper categorías. Modificar}\strut
\end{minipage}\tabularnewline
\midrule
\endhead
\begin{minipage}[t]{0.23\columnwidth}\raggedright\strut
\textbf{Requisitos asociados}\strut
\end{minipage} & \begin{minipage}[t]{0.71\columnwidth}\raggedright\strut
RF5.3.2 Modificar súper categoría\strut
\end{minipage}\tabularnewline
\begin{minipage}[t]{0.23\columnwidth}\raggedright\strut
\textbf{Descripción}\strut
\end{minipage} & \begin{minipage}[t]{0.71\columnwidth}\raggedright\strut
Se podrá modificar la súper categoría.
\strut
\end{minipage}\tabularnewline
\begin{minipage}[t]{0.23\columnwidth}\raggedright\strut
\textbf{Precondición}\strut
\end{minipage} & \begin{minipage}[t]{0.71\columnwidth}\raggedright\strut
Estar registrado en la aplicación como Administrador/Editor o Súper administrador.\strut
\end{minipage}\tabularnewline
\begin{minipage}[t]{0.23\columnwidth}\raggedright\strut
\textbf{Acciones}\strut
\end{minipage} & \begin{minipage}[t]{0.71\columnwidth}\raggedright\strut
\begin{enumerate}
\def\labelenumi{\arabic{enumi}.}
\tightlist
\item
Pinchar en el menú situado a la izquierda de la pantalla, en el
enlace a Gestión de datos.
\item
Seleccionar súper categorías.
\item
El sistema mostrará las súper categorías almacenadas en el sistema.
\item
Seleccionar la súper categoría que se desea modificar.
\item
El sistema enviará a una página con un formulario para modificar
la súper categoría y asignarle o quitarle categorías.
\item
Rellenar los datos que se desean modificar
\item
Hacer click en modificar.
\end{enumerate}\strut
\end{minipage}\tabularnewline
\begin{minipage}[t]{0.23\columnwidth}\raggedright\strut
\textbf{Postcondición}\strut
\end{minipage} & \begin{minipage}[t]{0.71\columnwidth}\raggedright\strut
Se modificará la súper categoría seleccionada.\strut
\end{minipage}\tabularnewline
\begin{minipage}[t]{0.23\columnwidth}\raggedright\strut
\textbf{Excepciones}\strut
\end{minipage} & \begin{minipage}[t]{0.71\columnwidth}\raggedright\strut
Ninguna. \strut
\end{minipage}\tabularnewline
\begin{minipage}[t]{0.23\columnwidth}\raggedright\strut
\textbf{Importancia}\strut
\end{minipage} & \begin{minipage}[t]{0.71\columnwidth}\raggedright\strut
Alta\strut
\end{minipage}\tabularnewline
\bottomrule
\caption{CU-19 Gestión de súper categorías. Modificar}
\end{longtable}

\newpage
\begin{longtable}[H]{@{}ll@{}}
\toprule
\begin{minipage}[b]{0.23\columnwidth}\raggedright\strut
\textbf{CU-20}\strut
\end{minipage} & \begin{minipage}[b]{0.71\columnwidth}\raggedright\strut
\textbf{Gestión de súper categorías. Borrar}\strut
\end{minipage}\tabularnewline
\midrule
\endhead
\begin{minipage}[t]{0.23\columnwidth}\raggedright\strut
\textbf{Requisitos asociados}\strut
\end{minipage} & \begin{minipage}[t]{0.71\columnwidth}\raggedright\strut
RF5.3.3 Borrar súper categoría\strut
\end{minipage}\tabularnewline
\begin{minipage}[t]{0.23\columnwidth}\raggedright\strut
\textbf{Descripción}\strut
\end{minipage} & \begin{minipage}[t]{0.71\columnwidth}\raggedright\strut
Se podrá borrar una súper categoría.
\strut
\end{minipage}\tabularnewline
\begin{minipage}[t]{0.23\columnwidth}\raggedright\strut
\textbf{Precondición}\strut
\end{minipage} & \begin{minipage}[t]{0.71\columnwidth}\raggedright\strut
Estar registrado en la aplicación como Administrador/Editor o Súper administrador.\strut
\end{minipage}\tabularnewline
\begin{minipage}[t]{0.23\columnwidth}\raggedright\strut
\textbf{Acciones}\strut
\end{minipage} & \begin{minipage}[t]{0.71\columnwidth}\raggedright\strut
\begin{enumerate}
\def\labelenumi{\arabic{enumi}.}
\tightlist
\item
Pinchar en el menú situado a la izquierda de la pantalla, en el
enlace a Gestión de datos.
\item
Seleccionar súper categorías.
\item
El sistema mostrará las súper categorías almacenadas en el sistema.
\item
Seleccionar la súper categoría que se desea borrar.
\end{enumerate}\strut
\end{minipage}\tabularnewline
\begin{minipage}[t]{0.23\columnwidth}\raggedright\strut
\textbf{Postcondición}\strut
\end{minipage} & \begin{minipage}[t]{0.71\columnwidth}\raggedright\strut
Se borrará la súper categoría seleccionada.\strut
\end{minipage}\tabularnewline
\begin{minipage}[t]{0.23\columnwidth}\raggedright\strut
\textbf{Excepciones}\strut
\end{minipage} & \begin{minipage}[t]{0.71\columnwidth}\raggedright\strut
Ninguna. \strut
\end{minipage}\tabularnewline
\begin{minipage}[t]{0.23\columnwidth}\raggedright\strut
\textbf{Importancia}\strut
\end{minipage} & \begin{minipage}[t]{0.71\columnwidth}\raggedright\strut
Alta\strut
\end{minipage}\tabularnewline
\bottomrule
\caption{CU-20 Gestión de súper categorías. Borrar}
\end{longtable}

\begin{longtable}[H]{@{}ll@{}}
\toprule
\begin{minipage}[b]{0.23\columnwidth}\raggedright\strut
\textbf{CU-21}\strut
\end{minipage} & \begin{minipage}[b]{0.71\columnwidth}\raggedright\strut
\textbf{Gestión de datos: Ámbitos. Crear}\strut
\end{minipage}\tabularnewline
\midrule
\endhead
\begin{minipage}[t]{0.23\columnwidth}\raggedright\strut
\textbf{Requisitos asociados}\strut
\end{minipage} & \begin{minipage}[t]{0.71\columnwidth}\raggedright\strut
RF5.4.1 Crear ámbito.\strut
\end{minipage}\tabularnewline
\begin{minipage}[t]{0.23\columnwidth}\raggedright\strut
\textbf{Descripción}\strut
\end{minipage} & \begin{minipage}[t]{0.71\columnwidth}\raggedright\strut
Se creará un ámbito geográfico.
\strut
\end{minipage}\tabularnewline
\begin{minipage}[t]{0.23\columnwidth}\raggedright\strut
\textbf{Precondición}\strut
\end{minipage} & \begin{minipage}[t]{0.71\columnwidth}\raggedright\strut
Estar registrado en la aplicación como Administrador/Editor o Súper administrador.\strut
\end{minipage}\tabularnewline
\begin{minipage}[t]{0.23\columnwidth}\raggedright\strut
\textbf{Acciones}\strut
\end{minipage} & \begin{minipage}[t]{0.71\columnwidth}\raggedright\strut
\begin{enumerate}
\def\labelenumi{\arabic{enumi}.}
\tightlist
\item
Pinchar en el menú situado a la izquierda de la pantalla, en el
enlace a Gestión de datos.
\item
Seleccionar ámbitos geográficos.
\item
El sistema mostrará los ámbitos almacenados en el sistema.
\item
Hacer click en el botón Crear ámbito geográfico.
\item
El sistema enviará a una página con un formulario para crear el ámbito.
\item
Rellenar el formulario.
\item
Hacer click en Crear.
\end{enumerate}\strut
\end{minipage}\tabularnewline
\begin{minipage}[t]{0.23\columnwidth}\raggedright\strut
\textbf{Postcondición}\strut
\end{minipage} & \begin{minipage}[t]{0.71\columnwidth}\raggedright\strut
Se creará el nuevo ámbito.\strut
\end{minipage}\tabularnewline
\begin{minipage}[t]{0.23\columnwidth}\raggedright\strut
\textbf{Excepciones}\strut
\end{minipage} & \begin{minipage}[t]{0.71\columnwidth}\raggedright\strut
Ninguna. \strut
\end{minipage}\tabularnewline
\begin{minipage}[t]{0.23\columnwidth}\raggedright\strut
\textbf{Importancia}\strut
\end{minipage} & \begin{minipage}[t]{0.71\columnwidth}\raggedright\strut
Alta\strut
\end{minipage}\tabularnewline
\bottomrule
\caption{CU-21 Gestión de ámbitos. Crear}
\end{longtable}

\newpage
\begin{longtable}[H]{@{}ll@{}}
\toprule
\begin{minipage}[b]{0.23\columnwidth}\raggedright\strut
\textbf{CU-22}\strut
\end{minipage} & \begin{minipage}[b]{0.71\columnwidth}\raggedright\strut
\textbf{Gestión de datos:Ámbitos geográficos. Modificar}\strut
\end{minipage}\tabularnewline
\midrule
\endhead
\begin{minipage}[t]{0.23\columnwidth}\raggedright\strut
\textbf{Requisitos asociados}\strut
\end{minipage} & \begin{minipage}[t]{0.71\columnwidth}\raggedright\strut
RF5.4.2 Modificar categoría\strut
\end{minipage}\tabularnewline
\begin{minipage}[t]{0.23\columnwidth}\raggedright\strut
\textbf{Descripción}\strut
\end{minipage} & \begin{minipage}[t]{0.71\columnwidth}\raggedright\strut
Se podrá modificar los ámbitos.
\strut
\end{minipage}\tabularnewline
\begin{minipage}[t]{0.23\columnwidth}\raggedright\strut
\textbf{Precondición}\strut
\end{minipage} & \begin{minipage}[t]{0.71\columnwidth}\raggedright\strut
Estar registrado en la aplicación como Administrador/Editor o Súper administrador.\strut
\end{minipage}\tabularnewline
\begin{minipage}[t]{0.23\columnwidth}\raggedright\strut
\textbf{Acciones}\strut
\end{minipage} & \begin{minipage}[t]{0.71\columnwidth}\raggedright\strut
\begin{enumerate}
\def\labelenumi{\arabic{enumi}.}
\tightlist
\item
Pinchar en el menú situado a la izquierda de la pantalla, en el
enlace a Gestión de datos.
\item
Seleccionar ámbitos geográficos.
\item
El sistema mostrará los ámbitos geográficos almacenados en el sistema.
\item
Seleccionar el ámbito que se desea modificar.
\item
El sistema enviará a una página con un formulario.
\item
Rellenar los datos que se desean modificar.
\item
Hacer click en modificar.
\end{enumerate}\strut
\end{minipage}\tabularnewline
\begin{minipage}[t]{0.23\columnwidth}\raggedright\strut
\textbf{Postcondición}\strut
\end{minipage} & \begin{minipage}[t]{0.71\columnwidth}\raggedright\strut
Se modificará el ámbito seleccionada.\strut
\end{minipage}\tabularnewline
\begin{minipage}[t]{0.23\columnwidth}\raggedright\strut
\textbf{Excepciones}\strut
\end{minipage} & \begin{minipage}[t]{0.71\columnwidth}\raggedright\strut
Ninguna. \strut
\end{minipage}\tabularnewline
\begin{minipage}[t]{0.23\columnwidth}\raggedright\strut
\textbf{Importancia}\strut
\end{minipage} & \begin{minipage}[t]{0.71\columnwidth}\raggedright\strut
Alta\strut
\end{minipage}\tabularnewline
\bottomrule
\caption{CU-22 Gestión de ámbitos geográficos. Modificar}
\end{longtable}

\newpage
\begin{longtable}[t]{@{}ll@{}}
\toprule
\begin{minipage}[t]{0.23\columnwidth}\raggedright\strut
\textbf{CU-23}\strut
\end{minipage} & \begin{minipage}[b]{0.71\columnwidth}\raggedright\strut
\textbf{Gestión de ámbitos. Borrar}\strut
\end{minipage}\tabularnewline
\midrule
\endhead
\begin{minipage}[t]{0.23\columnwidth}\raggedright\strut
\textbf{Requisitos asociados}\strut
\end{minipage} & \begin{minipage}[t]{0.71\columnwidth}\raggedright\strut
RF5.4.3 Borrar ámbito\strut
\end{minipage}\tabularnewline
\begin{minipage}[t]{0.23\columnwidth}\raggedright\strut
\textbf{Descripción}\strut
\end{minipage} & \begin{minipage}[t]{0.71\columnwidth}\raggedright\strut
Se podrá borrar un ámbito geográfico.
\strut
\end{minipage}\tabularnewline
\begin{minipage}[t]{0.23\columnwidth}\raggedright\strut
\textbf{Precondición}\strut
\end{minipage} & \begin{minipage}[t]{0.71\columnwidth}\raggedright\strut
Estar registrado en la aplicación como Administrador/Editor o Súper administrador.\strut
\end{minipage}\tabularnewline
\begin{minipage}[t]{0.23\columnwidth}\raggedright\strut
\textbf{Acciones}\strut
\end{minipage} & \begin{minipage}[t]{0.71\columnwidth}\raggedright\strut
\begin{enumerate}
\def\labelenumi{\arabic{enumi}.}
\tightlist
\item
Pinchar en el menú situado a la izquierda de la pantalla, en el
enlace a Gestión de datos.
\item
Seleccionar ámbitos geográficos.
\item
El sistema mostrará los ámbitos almacenados en el sistema.
\item
Seleccionar el ámbito que se desea borrar.
\end{enumerate}\strut
\end{minipage}\tabularnewline
\begin{minipage}[t]{0.23\columnwidth}\raggedright\strut
\textbf{Postcondición}\strut
\end{minipage} & \begin{minipage}[t]{0.71\columnwidth}\raggedright\strut
Se borrará el ámbito geográfico.\strut
\end{minipage}\tabularnewline
\begin{minipage}[t]{0.23\columnwidth}\raggedright\strut
\textbf{Excepciones}\strut
\end{minipage} & \begin{minipage}[t]{0.71\columnwidth}\raggedright\strut
Ninguna. \strut
\end{minipage}\tabularnewline
\begin{minipage}[t]{0.23\columnwidth}\raggedright\strut
\textbf{Importancia}\strut
\end{minipage} & \begin{minipage}[t]{0.71\columnwidth}\raggedright\strut
Alta\strut
\end{minipage}\tabularnewline
\bottomrule
\caption{CU-23 Gestión de ámbitos geográficos. Borrar}
\end{longtable}

\newpage
\begin{longtable}[t]{@{}ll@{}}
\toprule
\begin{minipage}[t]{0.23\columnwidth}\raggedright\strut
\textbf{CU-24}\strut
\end{minipage} & \begin{minipage}[b]{0.71\columnwidth}\raggedright\strut
\textbf{Exportación de tablas a xls}\strut
\end{minipage}\tabularnewline
\midrule
\endhead
\begin{minipage}[t]{0.23\columnwidth}\raggedright\strut
\textbf{Requisitos asociados}\strut
\end{minipage} & \begin{minipage}[t]{0.71\columnwidth}\raggedright\strut
RF6 Exportación de tablas a xls\strut
\end{minipage}\tabularnewline
\begin{minipage}[t]{0.23\columnwidth}\raggedright\strut
\textbf{Descripción}\strut
\end{minipage} & \begin{minipage}[t]{0.71\columnwidth}\raggedright\strut
Se podrá exportar las tablas a formato xls.
\strut
\end{minipage}\tabularnewline
\begin{minipage}[t]{0.23\columnwidth}\raggedright\strut
\textbf{Precondición}\strut
\end{minipage} & \begin{minipage}[t]{0.71\columnwidth}\raggedright\strut
Ninguna.\strut
\end{minipage}\tabularnewline
\begin{minipage}[t]{0.23\columnwidth}\raggedright\strut
\textbf{Acciones}\strut
\end{minipage} & \begin{minipage}[t]{0.71\columnwidth}\raggedright\strut
\begin{enumerate}
\def\labelenumi{\arabic{enumi}.}
\tightlist
\item
Pinchar en el menú situado a la izquierda de la pantalla, en el
enlace a Tablas predefinidas.
\item
Seleccionar una tabla.
\item
Elegir los datos de la tabla que quieres exportar.
\item
Hacer click en Exportar a Excel.
\end{enumerate}\strut
\end{minipage}\tabularnewline
\begin{minipage}[t]{0.23\columnwidth}\raggedright\strut
\textbf{Postcondición}\strut
\end{minipage} & \begin{minipage}[t]{0.71\columnwidth}\raggedright\strut
Se descargará el archivo xls.\strut
\end{minipage}\tabularnewline
\begin{minipage}[t]{0.23\columnwidth}\raggedright\strut
\textbf{Excepciones}\strut
\end{minipage} & \begin{minipage}[t]{0.71\columnwidth}\raggedright\strut
Ninguna. \strut
\end{minipage}\tabularnewline
\begin{minipage}[t]{0.23\columnwidth}\raggedright\strut
\textbf{Importancia}\strut
\end{minipage} & \begin{minipage}[t]{0.71\columnwidth}\raggedright\strut
Media\strut
\end{minipage}\tabularnewline
\bottomrule
\caption{CU-24 Exportación de tablas a Excel}
\end{longtable}

\newpage
\begin{longtable}[t]{@{}ll@{}}
\toprule
\begin{minipage}[t]{0.23\columnwidth}\raggedright\strut
\textbf{CU-25}\strut
\end{minipage} & \begin{minipage}[b]{0.71\columnwidth}\raggedright\strut
\textbf{Gestión de usuarios: Modificar rol de los usuarios}\strut
\end{minipage}\tabularnewline
\midrule
\endhead
\begin{minipage}[t]{0.23\columnwidth}\raggedright\strut
\textbf{Requisitos asociados}\strut
\end{minipage} & \begin{minipage}[t]{0.71\columnwidth}\raggedright\strut
RF7.1 Modificar rol de los usuarios \strut
\end{minipage}\tabularnewline
\begin{minipage}[t]{0.23\columnwidth}\raggedright\strut
\textbf{Descripción}\strut
\end{minipage} & \begin{minipage}[t]{0.71\columnwidth}\raggedright\strut
Se podrá modificar el rol a editor ó a súper administrador.
\strut
\end{minipage}\tabularnewline
\begin{minipage}[t]{0.23\columnwidth}\raggedright\strut
\textbf{Precondición}\strut
\end{minipage} & \begin{minipage}[t]{0.71\columnwidth}\raggedright\strut
Tener privilegios de súper administrador.\strut
\end{minipage}\tabularnewline
\begin{minipage}[t]{0.23\columnwidth}\raggedright\strut
\textbf{Acciones}\strut
\end{minipage} & \begin{minipage}[t]{0.71\columnwidth}\raggedright\strut
\begin{enumerate}
\def\labelenumi{\arabic{enumi}.}
\tightlist
\item
Pinchar en el menú situado a la izquierda de la pantalla, en el
enlace a Administración de usuarios.
\item
Seleccionar un usuario para editar.
\item
El sistema nos dará a elegir entre los posibles roles.
\item
Hacer click en aceptar.
\end{enumerate}\strut
\end{minipage}\tabularnewline
\begin{minipage}[t]{0.23\columnwidth}\raggedright\strut
\textbf{Postcondición}\strut
\end{minipage} & \begin{minipage}[t]{0.71\columnwidth}\raggedright\strut
Se cambiará el rol del usuario seleccionado.\strut
\end{minipage}\tabularnewline
\begin{minipage}[t]{0.23\columnwidth}\raggedright\strut
\textbf{Excepciones}\strut
\end{minipage} & \begin{minipage}[t]{0.71\columnwidth}\raggedright\strut
Ninguna. \strut
\end{minipage}\tabularnewline
\begin{minipage}[t]{0.23\columnwidth}\raggedright\strut
\textbf{Importancia}\strut
\end{minipage} & \begin{minipage}[t]{0.71\columnwidth}\raggedright\strut
Alta\strut
\end{minipage}\tabularnewline
\bottomrule
\caption{CU-25 Modificar rol de los usuarios}
\end{longtable}

\newpage
\begin{longtable}[t]{@{}ll@{}}
\toprule
\begin{minipage}[t]{0.23\columnwidth}\raggedright\strut
\textbf{CU-26}\strut
\end{minipage} & \begin{minipage}[b]{0.71\columnwidth}\raggedright\strut
\textbf{Gestión de usuarios: Borrar usuario}\strut
\end{minipage}\tabularnewline
\midrule
\endhead
\begin{minipage}[t]{0.23\columnwidth}\raggedright\strut
\textbf{Requisitos asociados}\strut
\end{minipage} & \begin{minipage}[t]{0.71\columnwidth}\raggedright\strut
RF7.2 Borrar usuario. \strut
\end{minipage}\tabularnewline
\begin{minipage}[t]{0.23\columnwidth}\raggedright\strut
\textbf{Descripción}\strut
\end{minipage} & \begin{minipage}[t]{0.71\columnwidth}\raggedright\strut
Se podrá modificar el rol a editor ó a súper administrador.
\strut
\end{minipage}\tabularnewline
\begin{minipage}[t]{0.23\columnwidth}\raggedright\strut
\textbf{Precondición}\strut
\end{minipage} & \begin{minipage}[t]{0.71\columnwidth}\raggedright\strut
Tener privilegios de súper administrador.\strut
\end{minipage}\tabularnewline
\begin{minipage}[t]{0.23\columnwidth}\raggedright\strut
\textbf{Acciones}\strut
\end{minipage} & \begin{minipage}[t]{0.71\columnwidth}\raggedright\strut
\begin{enumerate}
\def\labelenumi{\arabic{enumi}.}
\tightlist
\item
Pinchar en el menú situado a la izquierda de la pantalla, en el
enlace a Administración de usuarios.
\item
Seleccionar un usuario para borrar.
\item
Hacer click en aceptar.
\end{enumerate}\strut
\end{minipage}\tabularnewline
\begin{minipage}[t]{0.23\columnwidth}\raggedright\strut
\textbf{Postcondición}\strut
\end{minipage} & \begin{minipage}[t]{0.71\columnwidth}\raggedright\strut
Se borrará el usuario seleccionado.\strut
\end{minipage}\tabularnewline
\begin{minipage}[t]{0.23\columnwidth}\raggedright\strut
\textbf{Excepciones}\strut
\end{minipage} & \begin{minipage}[t]{0.71\columnwidth}\raggedright\strut
Ninguna. \strut
\end{minipage}\tabularnewline
\begin{minipage}[t]{0.23\columnwidth}\raggedright\strut
\textbf{Importancia}\strut
\end{minipage} & \begin{minipage}[t]{0.71\columnwidth}\raggedright\strut
Alta\strut
\end{minipage}\tabularnewline
\bottomrule
\caption{CU-26 Borrar usuarios}
\end{longtable}

\newpage
\begin{longtable}[t]{@{}ll@{}}
\toprule
\begin{minipage}[t]{0.23\columnwidth}\raggedright\strut
\textbf{CU-27}\strut
\end{minipage} & \begin{minipage}[b]{0.71\columnwidth}\raggedright\strut
\textbf{Gestión de usuarios:  Gestión de peticiones de nuevos usuarios}\strut
\end{minipage}\tabularnewline
\midrule
\endhead
\begin{minipage}[t]{0.23\columnwidth}\raggedright\strut
\textbf{Requisitos asociados}\strut
\end{minipage} & \begin{minipage}[t]{0.71\columnwidth}\raggedright\strut
RF7.3  Gestión de peticiones de nuevos usuarios. \strut
\end{minipage}\tabularnewline
\begin{minipage}[t]{0.23\columnwidth}\raggedright\strut
\textbf{Descripción}\strut
\end{minipage} & \begin{minipage}[t]{0.71\columnwidth}\raggedright\strut
Los usuarios que quieran solicitar el registro en la página deben rellenar un
formulario para que el súper administrador decida si acepta o declina esta
solicitud. Esta solicitud podrá ser aceptada o declinada.
\strut
\end{minipage}\tabularnewline
\begin{minipage}[t]{0.23\columnwidth}\raggedright\strut
\textbf{Precondición}\strut
\end{minipage} & \begin{minipage}[t]{0.71\columnwidth}\raggedright\strut
Tener privilegios de súper administrador y que haya solicitudes de registro.\strut
\end{minipage}\tabularnewline
\begin{minipage}[t]{0.23\columnwidth}\raggedright\strut
\textbf{Acciones}\strut
\end{minipage} & \begin{minipage}[t]{0.71\columnwidth}\raggedright\strut
\begin{enumerate}
\def\labelenumi{\arabic{enumi}.}
\tightlist
\item
Pinchar en el menú situado a la izquierda de la pantalla, en el
enlace a Administración de usuarios.
\item
El sistema mostrará las peticiones de registro de los usuarios.
\item
\begin{enumerate}
    \item El súper administrador acepta la petición.
    \item El súper administrador declina la petición.
\end{enumerate}
\end{enumerate}\strut
\end{minipage}\tabularnewline
\begin{minipage}[t]{0.23\columnwidth}\raggedright\strut
\textbf{Postcondición}\strut
\end{minipage} & \begin{minipage}[t]{0.71\columnwidth}\raggedright\strut
Se acepta la petición de registro o se deniega.\strut
\end{minipage}\tabularnewline
\begin{minipage}[t]{0.23\columnwidth}\raggedright\strut
\textbf{Excepciones}\strut
\end{minipage} & \begin{minipage}[t]{0.71\columnwidth}\raggedright\strut
Ninguna. \strut
\end{minipage}\tabularnewline
\begin{minipage}[t]{0.23\columnwidth}\raggedright\strut
\textbf{Importancia}\strut
\end{minipage} & \begin{minipage}[t]{0.71\columnwidth}\raggedright\strut
Alta\strut
\end{minipage}\tabularnewline
\bottomrule
\caption{CU-27 Borrar usuarios}
\end{longtable}

\newpage
\begin{longtable}[t]{@{}ll@{}}
\toprule
\begin{minipage}[t]{0.23\columnwidth}\raggedright\strut
\textbf{CU-28}\strut
\end{minipage} & \begin{minipage}[b]{0.71\columnwidth}\raggedright\strut
\textbf{Gestión de datos desde el INE: Introducción de datos.}\strut
\end{minipage}\tabularnewline
\midrule
\endhead
\begin{minipage}[t]{0.23\columnwidth}\raggedright\strut
\textbf{Requisitos asociados}\strut
\end{minipage} & \begin{minipage}[t]{0.71\columnwidth}\raggedright\strut
RF8.1  Introducción de datos. \strut
\end{minipage}\tabularnewline
\begin{minipage}[t]{0.23\columnwidth}\raggedright\strut
\textbf{Descripción}\strut
\end{minipage} & \begin{minipage}[t]{0.71\columnwidth}\raggedright\strut
El administrador o el súper administrador pueden introducir datos a las tablas de la aplicación a partir de la dirección web de una página del INE.

\strut
\end{minipage}\tabularnewline
\begin{minipage}[t]{0.23\columnwidth}\raggedright\strut
\textbf{Precondición}\strut
\end{minipage} & \begin{minipage}[t]{0.71\columnwidth}\raggedright\strut
Tener privilegios de súper administrador o administrador/editor.\strut
\end{minipage}\tabularnewline
\begin{minipage}[t]{0.23\columnwidth}\raggedright\strut
\textbf{Acciones}\strut
\end{minipage} & \begin{minipage}[t]{0.71\columnwidth}\raggedright\strut
\begin{enumerate}
\def\labelenumi{\arabic{enumi}.}
\tightlist
\item
Pinchar en el menú situado a la izquierda de la pantalla, en el
enlace a Gestión de datos del INE.
\item
Se introducirá la dirección de la página del INE de la que queremos sacar la información.
\item
El sistema clasificará esa información y dará a elegir al usuario las variables y años de los datos que quiere introducir en la aplicación.
\item
Se muestran los datos elegidos y se selecciona en qué tabla van a introducirse y a qué categoría y ámbito van asociados.
\item 
Se selecciona Aceptar.
\end{enumerate}\strut
\end{minipage}\tabularnewline
\begin{minipage}[t]{0.23\columnwidth}\raggedright\strut
\textbf{Postcondición}\strut
\end{minipage} & \begin{minipage}[t]{0.71\columnwidth}\raggedright\strut
Se introducen los valores en la base de datos.\strut
\end{minipage}\tabularnewline
\begin{minipage}[t]{0.23\columnwidth}\raggedright\strut
\textbf{Excepciones}\strut
\end{minipage} & \begin{minipage}[t]{0.71\columnwidth}\raggedright\strut
Si es un conjunto de datos que la aplicación no puede soportar, da un aviso. \strut
\end{minipage}\tabularnewline
\begin{minipage}[t]{0.23\columnwidth}\raggedright\strut
\textbf{Importancia}\strut
\end{minipage} & \begin{minipage}[t]{0.71\columnwidth}\raggedright\strut
Alta\strut
\end{minipage}\tabularnewline
\bottomrule
\caption{CU-28 Gestión de datos del INE: Introducción de datos}
\end{longtable}

\newpage
\begin{longtable}[t]{@{}ll@{}}
\toprule
\begin{minipage}[t]{0.23\columnwidth}\raggedright\strut
\textbf{CU-29}\strut
\end{minipage} & \begin{minipage}[b]{0.71\columnwidth}\raggedright\strut
\textbf{Gestión de datos desde el INE: Actualización de datos.}\strut
\end{minipage}\tabularnewline
\midrule
\endhead
\begin{minipage}[t]{0.23\columnwidth}\raggedright\strut
\textbf{Requisitos asociados}\strut
\end{minipage} & \begin{minipage}[t]{0.71\columnwidth}\raggedright\strut
RF8.2  Actualización de datos. \strut
\end{minipage}\tabularnewline
\begin{minipage}[t]{0.23\columnwidth}\raggedright\strut
\textbf{Descripción}\strut
\end{minipage} & \begin{minipage}[t]{0.71\columnwidth}\raggedright\strut
 El administrador o el súper administrador pueden actualizar las tablas que contengan datos que provengan del INE con información nueva.
\strut
\end{minipage}\tabularnewline
\begin{minipage}[t]{0.23\columnwidth}\raggedright\strut
\textbf{Precondición}\strut
\end{minipage} & \begin{minipage}[t]{0.71\columnwidth}\raggedright\strut
Tener privilegios de súper administrador o administrador/editor.\strut
\end{minipage}\tabularnewline
\begin{minipage}[t]{0.23\columnwidth}\raggedright\strut
\textbf{Acciones}\strut
\end{minipage} & \begin{minipage}[t]{0.71\columnwidth}\raggedright\strut
\begin{enumerate}
\def\labelenumi{\arabic{enumi}.}
\tightlist
\item
Pinchar en el menú situado a la izquierda de la pantalla, en el
enlace a Gestión de datos del INE.
\item
Se seleccionará la opción Actualizar tablas.
\item
El sistema comprobará qué tablas tienen datos del INE asociados, mirará si hay datos nuevos en la página del INE y actualizará las tablas.
\item
Se mostrarán los nombres de las tablas actualizadas.
\end{enumerate}\strut
\end{minipage}\tabularnewline
\begin{minipage}[t]{0.23\columnwidth}\raggedright\strut
\textbf{Postcondición}\strut
\end{minipage} & \begin{minipage}[t]{0.71\columnwidth}\raggedright\strut
Se actualizan los valores en la base de datos.\strut
\end{minipage}\tabularnewline
\begin{minipage}[t]{0.23\columnwidth}\raggedright\strut
\textbf{Excepciones}\strut
\end{minipage} & \begin{minipage}[t]{0.71\columnwidth}\raggedright\strut
Ninguna. \strut
\end{minipage}\tabularnewline
\begin{minipage}[t]{0.23\columnwidth}\raggedright\strut
\textbf{Importancia}\strut
\end{minipage} & \begin{minipage}[t]{0.71\columnwidth}\raggedright\strut
Alta\strut
\end{minipage}\tabularnewline
\bottomrule
\caption{CU-29 Gestión de datos del INE: Actualización de datos}
\end{longtable}


\newpage
\begin{longtable}[t]{@{}ll@{}}
\toprule
\begin{minipage}[t]{0.23\columnwidth}\raggedright\strut
\textbf{CU-30}\strut
\end{minipage} & \begin{minipage}[b]{0.71\columnwidth}\raggedright\strut
\textbf{Predicción de datos}\strut
\end{minipage}\tabularnewline
\midrule
\endhead
\begin{minipage}[t]{0.23\columnwidth}\raggedright\strut
\textbf{Requisitos asociados}\strut
\end{minipage} & \begin{minipage}[t]{0.71\columnwidth}\raggedright\strut
RF9 Predicción de datos. \strut
\end{minipage}\tabularnewline
\begin{minipage}[t]{0.23\columnwidth}\raggedright\strut
\textbf{Descripción}\strut
\end{minipage} & \begin{minipage}[t]{0.71\columnwidth}\raggedright\strut
Todos los usuarios pueden acceder a una predicción del siguiente año de las variables de la aplicación.
\strut
\end{minipage}\tabularnewline
\begin{minipage}[t]{0.23\columnwidth}\raggedright\strut
\textbf{Precondición}\strut
\end{minipage} & \begin{minipage}[t]{0.71\columnwidth}\raggedright\strut
Ninguna.\strut
\end{minipage}\tabularnewline
\begin{minipage}[t]{0.23\columnwidth}\raggedright\strut
\textbf{Acciones}\strut
\end{minipage} & \begin{minipage}[t]{0.71\columnwidth}\raggedright\strut
\begin{enumerate}
\def\labelenumi{\arabic{enumi}.}
\tightlist
\item
Pinchar en el menú situado a la izquierda de la pantalla, en el
enlace a Predicción de datos.
\item
Se muestran las tablas de la aplicación.
\item
Se seleccionará la tabla de la que queramos ver la predicción.
\item
Ahora se seleccionará la categoría y el ámbito para la predicción.
\item
Se mostrarán las predicciones en forma de tabla y de gráfico.
\end{enumerate}\strut
\end{minipage}\tabularnewline
\begin{minipage}[t]{0.23\columnwidth}\raggedright\strut
\textbf{Postcondición}\strut
\end{minipage} & \begin{minipage}[t]{0.71\columnwidth}\raggedright\strut
Se muestran las predicciones.\strut
\end{minipage}\tabularnewline
\begin{minipage}[t]{0.23\columnwidth}\raggedright\strut
\textbf{Excepciones}\strut
\end{minipage} & \begin{minipage}[t]{0.71\columnwidth}\raggedright\strut
Ninguna. \strut
\end{minipage}\tabularnewline
\begin{minipage}[t]{0.23\columnwidth}\raggedright\strut
\textbf{Importancia}\strut
\end{minipage} & \begin{minipage}[t]{0.71\columnwidth}\raggedright\strut
Baja\strut
\end{minipage}\tabularnewline
\bottomrule
\caption{CU-30 Predicción de datos}
\end{longtable}

\newpage
\begin{longtable}[t]{@{}ll@{}}
\toprule
\begin{minipage}[t]{0.23\columnwidth}\raggedright\strut
\textbf{CU-31}\strut
\end{minipage} & \begin{minipage}[b]{0.71\columnwidth}\raggedright\strut
\textbf{Ayuda de la aplicación}\strut
\end{minipage}\tabularnewline
\midrule
\endhead
\begin{minipage}[t]{0.23\columnwidth}\raggedright\strut
\textbf{Requisitos asociados}\strut
\end{minipage} & \begin{minipage}[t]{0.71\columnwidth}\raggedright\strut
RF10 Ayuda de la aplicación. \strut
\end{minipage}\tabularnewline
\begin{minipage}[t]{0.23\columnwidth}\raggedright\strut
\textbf{Descripción}\strut
\end{minipage} & \begin{minipage}[t]{0.71\columnwidth}\raggedright\strut
El usuario debe poder obtener ayuda sobre el uso de las funcionalidades de la aplicación.
\strut
\end{minipage}\tabularnewline
\begin{minipage}[t]{0.23\columnwidth}\raggedright\strut
\textbf{Precondición}\strut
\end{minipage} & \begin{minipage}[t]{0.71\columnwidth}\raggedright\strut
Ninguna.\strut
\end{minipage}\tabularnewline
\begin{minipage}[t]{0.23\columnwidth}\raggedright\strut
\textbf{Acciones}\strut
\end{minipage} & \begin{minipage}[t]{0.71\columnwidth}\raggedright\strut
\begin{enumerate}
\def\labelenumi{\arabic{enumi}.}
\tightlist
\item
Pinchar en el menú situado a la izquierda de la pantalla, en el
enlace a Ayuda.
\item
Se visualizará la ayuda.
\end{enumerate}\strut
\end{minipage}\tabularnewline
\begin{minipage}[t]{0.23\columnwidth}\raggedright\strut
\textbf{Postcondición}\strut
\end{minipage} & \begin{minipage}[t]{0.71\columnwidth}\raggedright\strut
Se muestra la ayuda de la aplicación.\strut
\end{minipage}\tabularnewline
\begin{minipage}[t]{0.23\columnwidth}\raggedright\strut
\textbf{Excepciones}\strut
\end{minipage} & \begin{minipage}[t]{0.71\columnwidth}\raggedright\strut
Ninguna. \strut
\end{minipage}\tabularnewline
\begin{minipage}[t]{0.23\columnwidth}\raggedright\strut
\textbf{Importancia}\strut
\end{minipage} & \begin{minipage}[t]{0.71\columnwidth}\raggedright\strut
Baja\strut
\end{minipage}\tabularnewline
\bottomrule
\caption{CU-31 Ayuda de la aplicación}
\end{longtable}