\capitulo{6}{Trabajos relacionados}
\section{Api de datos abiertos de la junta de Castilla y León}
El portal de Datos Abiertos se enmarca en el proyecto de Gobierno Abierto de la Junta de Castilla y León por el que se pone en marcha el Modelo de Gobierno Abierto de la Junta de Castilla y León junto con la información de transparencia, el espacio de participación ciudadana   y la presencia en redes sociales entre otras actuaciones.\\
Este portal tiene como objetivos aumentar la transparencia, proporcionando mayor información sobre la actividad de la Junta de Castilla y León y conseguir la participación y colaboración de los ciudadanos y empresas, a través de la interlocución con los mismos, de manera que el intercambio de conocimiento y experiencias permita el avance conjunto de la iniciativa pública y privada.\cite{datosab}\\
\imagen{imagenes/datosjcyl}{Algunos conjuntos de datos de la aplicación}
\section{Instituto nacional de estadística}
La página web del INE cuenta con un funcionamiento muy parecido al que se ha desarrollado en esta aplicación.\\
En ella también se muestra la información de las variables en tablas y en gráficos.\\
\imagen{imagenes/tablaine}{Ejemplo de una tabla del INE}
\imagen{imagenes/graficoine}{Ejemplo de un gráfico del INE}
La página web no va destinada directamente a la creación de tablas y gráficos, sino a dar a conocer que es el INE, sus métodos y proyectos y sus productos o servicios.\\
El INE utiliza varios programas informáticos:
\begin{description}
    \item [Sorolla] Es un sistema de apoyo a la gestión económica de los centros gestores públicos. Entre otras herramientas, se destaca un módulo de elaboración de informes de ejecución presupuestaria (AVANCE). Cuenta con un módulo para elaboración de documentación tributaria relativa a los pagos realizados a personas físicas y jurídicas. Por lo que, ayuda en la elaboración de modelos estadísticos predictivos, sobre la base de la elaboración de los presupuestos.
    \item [Greco] Es un software para el análisis y tratamiento estadístico relativo al turismo, referido a alojamientos, hoteles, casas rurales, etc.
    \item [Celec] Software que se encarga de analizar datos estadísticos relativos al censo de población, extenso y completo. Es el mas utilizado por el INE.
    \item [Ocr] Programa que se encarga de analizar los nacimientos, defunciones, matrimonios y partos de la población. Realiza un estudio de las variaciones de la población española.
    \item [Ida padrón] Software para el análisis y tratamiento de los datos de empadronamiento, se analizan mediante un censo de población junto con un censo de viviendas con una periodicidad de diez años.
    \item [Epf] Aplicación que gestiona las encuestas que hacen referencia a los presupuestos familiares.
\end{description}
