\capitulo{1}{Introducción}
Todos los años, el departamento de economía aplicada de la Universidad de Burgos realiza un boletín exhaustivo sobre la coyuntura económica en el ámbito burgalés (\href{https://www.ubu.es/departamento-de-economia-aplicada/investigacion-research/grupos-de-investigacion-research-groups/equipo-de-coyuntura-economica-de-burgos}{Enlace a algunos boletines}).\\
Para ello recopila datos de distintas fuentes: Instituto Nacional de estadística (INE), Banco Nacional de España, Eurostat...\\
La aplicación se creó en 2017 para usarse como una herramienta que permita organizar y tratar dichos datos, así como almacenarlos y facilitar el trabajo al departamento que los recoge.\\
Mi trabajo consiste en mejorar dicha aplicación y ampliar su funcionalidad.\\ 
El objetivo de la realización de los boletines de coyuntura económica es conocer el desarrollo de la economía del ámbito estudiado, mediante el análisis de información económica y su divulgación a un amplio público de empresas, profesionales y particulares.\\
Estos boletines los crea el Equipo multidisciplinar de Coyuntura radicado en la Facultad de Ciencias Económicas y Empresariales de la Universidad de Burgos. En virtud del Convenio Marco de Colaboración firmado por la Universidad de Burgos y la actual Caja Viva Caja Rural.\\
Este equipo multidisciplinar integrado por 16 profesores, de los Departamentos de Economía Aplicada, Economía y Administración de Empresas y Derecho analiza la evolución económica coyuntural de la provincia de Burgos.\\
Aunque el cliente final de la aplicación es este equipo, también se permitirá el acceso a invitados para que puedan ver la información recogida por lo que existe un sistema de roles con distintos permisos que se explicarán más adelante.\\
La principal funcionalidad de la aplicación consiste en la introducción de datos estadísticos de la coyuntura económica dependientes de una categoría específica, un ámbito geográfico y un
año.\\
A partir de estos datos se generarán tablas que podrán ser filtradas y mostradas al usuario de la manera que a éste le resulte mas cómoda y utilizarlas para realizar el boletín.\\
La aplicación también genera gráficos que muestran los datos de una forma sencilla e intuitiva.\\
Los invitados que accedan a la aplicación deberán tener un rol solo de lectura, por lo que sólo podrán filtrar y visualizar las tablas y gráficos de las variables económicas que están almacenadas en la base de datos de la aplicación.\\
Los encargados de introducir datos estadísticos, modificarlos o borrar los que estén obsoletos o erróneos, serán los administradores de la aplicación.\\
La aplicación proporciona una interfaz web que permite la entrada y el almacenamiento de variables e incluye herramientas para la visualización de datos entre otras funcionalidades.\\
Mi aportación a esta aplicación se basa en realizar algunas mejoras, corregir errores y añadir nuevas funcionalidades como la de introducir datos de forma automatizada desde el INE o la predicción de datos.\\


