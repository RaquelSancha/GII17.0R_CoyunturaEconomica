\capitulo{5}{Aspectos relevantes del desarrollo del proyecto}
En este apartado se van a recoger los aspectos más importantes del desarrollo del proyecto. Desde cuestiones del desarrollo del proyecto, hasta los numerosos problemas a los que hubo que enfrentarse y cómo se solucionaron.
\section{Inicio del proyecto: Aprendizaje}
En esta fase, probablemente la más costosa a nivel personal, detallaré los conceptos que me hicieron falta aprender para poder empezar mi trabajo.\\
\subsection{Aprendizaje de Laravel} 
Laravel tiene un sistema de carpetas y archivos de configuración muy extenso que al principio cuesta entender pero por suerte existen multitud de tutoriales en forma de vídeos o de artículos que explican como iniciarse en Laravel.\\
Si no se tiene ningún conocimiento previo sobre este framework no recomendaría mirar la documentación de la página oficial ya que, aunque es muy completa, no considero que ofrezca una visión general sobre el funcionamiento del mismo sino información muy específica que no sirve al usuario si no conoce primero la dinámica general de la herramienta.\\
Tutoriales que seguí:
\begin{itemize}
    \item \textit{Laravel PHP Framework Tutorial}(freeCodeCamp)\cite{TutorialLaravel}
    \item \textit{Curso Laravel}(pildorasinformaticas)\cite{TutorialLaravel2}
    \item \textit{Ejemplo de un proyecto en laravel}(Julio Yáñez Novo)\cite{PrimerProyecto}
\end{itemize}
\subsection{Aprendizaje de PHP} 
Este lenguaje no se enseña en la carrera y tampoco había trabajado con él antes así que tuve que aprenderlo desde cero. No me resultó tan difícil como aprender a entender proyectos con Laravel así que esta parte me supuso menos trabajo.\\
Las herramientas que usé para familiarizarme con este lenguaje fueron:
\begin{itemize}
    \item \textit{Tutorial de PHP para principiantes}(ionos)\cite{TutorialPHP}
    \item \textit{Curso php}(tutorialesya)\cite{CursoPHP} 
\end{itemize}
\section{Gestión de los errores}
En la aplicación original existían varios errores que hubo que solucionar. 
\begin{description}
    \item [Exportación de datos a excel]. Existía un botón para exportar los datos de la tabla elegida a Excel pero no funcionaba. Para conseguir que el botón cumpliera la funcionalidad prometida he usado la biblioteca Laravel Excel (explicada anteriormente en el apartado "Técnicas y herramientas")
    \item[Administración de usuarios] Esta sección daba un error cuando se intentaba acceder a ella.\\
    Para entender esta sección hace falta conocer el sistema de usuarios que usa la aplicación:
    \begin{description}
        \item [Súper administrador] Las personas con este rol pueden controlar los roles de los demás usuarios y modificarlos.
        Además se les permite aceptar o declinar las peticiones de registro en la aplicación a nuevos usuarios. También tienen acceso al resto las funcionalidades de la aplicación.
        \item [Administrador] Este tipo de usuarios tienen acceso a todas las partes de la aplicación menos a la de la administración de usuarios, es decir, no pueden aceptar nuevos usuarios ni cambiar roles.
        \item [Invitado] Este tipo de usuarios no está registrado en la aplicación. Sólo tiene acceso a las tablas para su visualización. Para poder acceder a las demás funcionalidades deberá realizar una petición de registro.
    \end{description}
    \imagen{imagenes/Administracionusuarios}{Sección Administración de usuarios}
    Desde esta pantalla, la cual solo se puede acceder si tienes privilegios de súper administrador, se permite aceptar las solicitudes pendientes de creación de una cuenta así como editar y borrar otros perfiles.
    \imagen{imagenes/EditarUsuario}{Editar datos de un usuario}
    También se creó esta pantalla para permitir al usuario editar sus datos.
\end{description}
\section{Ampliación de funcionalidades}
Después de subsanar los errores que existían en la aplicación, se decidió implementar más funcionalidades.\\
\subsection{Introducción de datos desde una base de datos externa}
Para realizar el informe de la coyuntura económica de Burgos se usan múltiples fuentes así que se pensó en una forma de sacar la información de una manera más sencilla para el usuario.\\
Las bases de datos abiertos más utilizadas son: el Instituto Nacional de Estadística, Eurostat y los datos abiertos de la Junta de Castilla y León. En esta sección se hablará de cada una de ellas y se explicará su manera de extraer datos concluyendo con la justificación de la plataforma elegida.
\subsubsection{Instituto nacional de estadística}
Una fuente de la que suelen sacar los datos para el análisis de la coyuntura económica es el Instituto Nacional de Estadística (INE). Por ello, para facilitar la extracción de los datos, se pensó en la posibilidad de permitir seleccionar la información de esta página y que la aplicación la insertara automáticamente en la base de datos.\\
El catálogo de datos de esta plataforma proviene de dos fuentes: la base de datos de difusión Tempus3 y el repositorio de ficheros PC-Axis.\\
La información se distribuye en tablas en las que se permite filtrar los datos según sus características.
 \imagen{imagenes/Ine}{Ejemplo de una tabla del INE}
Una vez se seleccionen los valores que se quieran consultar, los datos aparecen en forma de tablas o distribuidos en un gráfico.
\imagen{imagenes/tablaine2}{Datos de una tabla del INE}
Además de esta manera de visualizar los datos, el Instituto Nacional de Estadística proporciona un servicio que permite descargar la información en formato JSON. Para ello debemos crear una petición en forma de url.\\
\imagen{imagenes/datosine}{Salida de los datos en formato JSON}
 La url para la petición de los datos tiene el formato que aparece en la imagen. 
 \imagen{imagenes/urlJSON}{Estructura de la url}
Los campos que aparecen entre llaves, \{ \}, son obligatorios.\cite{ine:urljson}\\
Los campos que aparecen entre corchetes, [ ], son opcionales y cambian en relación a la función considerada.\\
Descripción de cada uno de ellos
\begin{description}
	\item [idioma] ES para español e EN para inglés. Por defecto está puesto el español.\\
	\item [función] Funciones implementadas en el sistema para poder realizar diferentes tipos de consulta en función del tipo de fuente, Tempus3 o PC-Axis, y del elemento que se quiere obtener.
    \item [inputs] Identificadores de los elementos de entrada de las funciones. Estos inputs varían en base al repositorio del que se extraen los datos.
    \item [parámetros] Los parámetros en la URL se establecen a partir del símbolo ?.\\
    Cuando haya más de un parámetro, el símbolo \& se utiliza como separador.\\
    No todas las funciones admiten todos los parámetros posibles. Por ello haremos una clasificación para explicarlos mejor:  
     \begin{enumerate}
        \item Parámetros comunes a todas las funciones
            \begin{description}
            \item [page] Si hay más de 500 elementos, la consulta se divide en páginas. Esta opción nos permite seleccionar la página que queremos visualizar.
            \item [download] Para descargar el fichero JSON.
            \item [det] Este parámetro da más detalles de la información mostrada.
            \item [tip] Cambia la forma de mostrar la información.
            \end{description}
        \item Parámetros para la petición de datos
            \begin{description}
                \item [date] Filtra los datos por fecha; fecha concreta, lista o rango de fechas.
                \item [nult] Devuelve los últimos n datos. Ejemplo: nult=4 devuelve los 4 últimos datos.
            \end{description}
        \item Parámetros para la obtención de datos y metadatos en base al ámbito geográfico
            \begin{description}
                \item [geo] Con geo = 1 para provincias, municipios u otras desagregaciones y geo = 0 para datos nacionales.
            \end{description}
        \end{enumerate}
\end{description}
\subsubsection{Eurostat}
Eurostat es una base de datos que recoge información a nivel europeo. Tiene un servicio de datos en JSON y en Unicode mediante el cual permite descargar los archivos disponibles en estos dos formatos a través de una petición REST.
\imagen{imagenes/eurostat}{Estructura de una petición REST de Eurostat}\cite{eurostat}
Como se puede ver la estructura es muy parecida a la petición de datos del INE, con una parte fija, otra parte que indica el formato, una que indica el lenguaje y finalmente el código que identifica los datos. También permite filtrar la información.\\
El inconveniente de este servicio es la forma de conseguir el identificador de los datos (\textit{datasetCode} en la imagen).
Los códigos de los conjuntos de datos se pueden encontrar ordenados por temas en la página de Eurostat
\imagen{imagenes/EurostatData}{Conjuntos de datos del Eurostat}
\subsubsection{Datos abiertos de la Junta de Castilla y León}
El portal de Datos Abiertos se enmarca en el proyecto de Gobierno Abierto de la Junta de Castilla y León por el que se pone en marcha el Modelo de Gobierno Abierto de la Junta de Castilla y León junto con la información de transparencia, el espacio de participación ciudadana   y la presencia en redes sociales entre otras actuaciones.\\
Este portal tiene como objetivos aumentar la transparencia, proporcionando mayor información sobre la actividad de la Junta de Castilla y León y conseguir la participación y colaboración de los ciudadanos y empresas, a través de la interlocución con los mismos, de manera que el intercambio de conocimiento y experiencias permita el avance conjunto de la iniciativa pública y privada.\cite{datosab}\\
\imagen{imagenes/apijcyl}{Búsqueda de datos desde la API}
Esta plataforma también permite descargar la información en formato JSON, CSV o Excel. Su funcionamiento es muy simple pero no permite automatizar la extracción de los datos como las bases de datos anteriores.
\imagen{imagenes/datosjcyl}{Algunos conjuntos de datos de la aplicación}
\subsubsection{Conclusión}
Para elegir qué plataforma usar para la extracción de datos se hizo una evaluación de sus características.
\tablaSmall{Comparativa de las plataformas de datos abiertos}{l c c c c}{comparativaplataformas}
{ \multicolumn{1}{l}{} & INE & Eurostat & Datos abiertos JCYL \\}{ 
Posee un sistema de petición de datos por url & X & X & \\
La petición de los datos puede automatizarse & X & & \\
Posee un gran catálogo de datos & X & X & X\\
Facilidad de uso & X & & X\\
Facilidad de extracción de datos & X & &\\
}
Habiendo sopesado las tres opciones, se eligió la plataforma del Instituto Nacional de Estadística por ser la más completa y sencilla de automatizar.
\subsection{Predicción de datos}
Otra funcionalidad interesante que se decidió añadir fue la posibilidad de predecir los datos de una variable.//
Para ello, se utilizaron una serie de algoritmos de predicción de datos. 

\section{Problemas}
A lo largo del desarrollo del proyecto se encontraron problemas que condicionaron el trabajo. En esta sección se expondrán y se explicarán cómo se solucionaron.
\subsection{Problemas de compatibilidad}
Este tipo de problemas tuvieron que ver con Composer.\\
Como ya se ha explicado, Composer es un gestor de dependencias del proyecto. El problema de usar este gestor es que al instalar nuevas librerías era muy frecuente que aparecieran errores de compatibilidad.
\imagen{imagenes/problemaComposer}{Captura de un problema típico al intentar instalar un paquete nuevo}
Para la resolución de estos problemas se han tenido que actualizar, borrar e incluso instalar versiones viejas de las dependencias problemáticas.
\subsection{Problemas en el despliegue del proyecto}
Para la producción del trabajo se han usado muchos servidores locales y on-line como Xampp, Wampp, Docker, etc y todos ellos han supuesto un reto en su correcta configuración.\\
\subsubsection{Xampp}
Xampp ha sido la herramienta más usada para probar el proyecto.\\
La mayoría de los problemas los tuve a la hora de configurar el host virtual.
\imagen{imagenes/vhostxamp}{Configuración de los hosts virtuales}
Al principio, la ruta no era resuelta correctamente por el navegador. Para la solución a este problema hizo falta cambiar el archivo de configuración del servidor de Xampp, y habilitar los hosts virtuales.\\
Además hubo que borrar la caché del navegador ya que cargaba la anterior configuración.
\subsubsection{Docker}
Para usar esta herramienta se usó el contenedor \textit{laradock} que tiene multitud de imágenes para proyectos de Laravel. El principal problema fue el servidor que utilizaba ya que en un principio se usó \textit{nginx} y no funcionaba correctamente. Después se cambió a \textit{apache}.\\
\imagen{imagenes/envlaradock}{Archivo de configuración .env de laradock}
El direccionamiento del proyecto se realiza desde el archivo \textit{.env}.\\
Laradock usa un direccionamiento virtual de los archivos así que esta parte también fue problemática.\\
\subsubsection{Heroku}
Para la presentación de la aplicación se decidió utilizar en Heroku en un principio, que es un hosting gratuito donde se puede alojar tu aplicación web. Hubo varios problemas al instalarlo ya que el proyecto tiene dos partes, la parte del código y la base de datos. Heroku ofrece una base de datos propia pero es de pago, además está en PostgreSQL y la del proyecto está en MySQL.\\
Para solucionar esto se consiguió conectar la aplicación a una base de datos externa totalmente gratuita llamada \href{https://www.db4free.net/}{db4free.net}.\\
Cuando se consiguió que todo funcionara correctamente, se notó la lentitud de la aplicación, pero al ser un hosting gratuito
es normal ya que no nos ofrecen mucha memoria.\\
La aplicación funciona, pero al tener tantos servicios con peticiones agotamos la memoria RAM que nos ofrece Heroku y el rendimiento de la aplicación no es el esperado.\\
Para poder tener un buen rendimiento, sería necesario usar la versión de pago de Heroku, mucho más potente.






